\section{Overview and Introduction}

\subsection{Document Overview}

This document defines the technical specifications, development requirements, software components, and ResIoT configurations of the \gls{IoT}-based Plant Monitoring System implemented during the course \textit{Sensor Networks}. The objective of this specification is to establish a clear and comprehensive reference for the design, implementation, and verification of a system that integrates long-range connectivity via \gls{LoRaWAN}.

\subsection{Project Introduction}

The main goal of Project 1 is to enable \gls{LoRaWAN} connectivity on an STM32-based \gls{IoT} node running the Zephyr \gls{RTOS}. The system is designed to monitor the environmental conditions and physiological state of a plant.

\subsection{Summary of the Work Done}

This section provides a consolidated overview of the work completed, which builds upon the sensor management hardware and software developed in the previous subject. The current work focuses on the integration of the \gls{LoRaWAN} stack and ResIoT data visualization.

\subsubsection{Phase 1: \acrshort{LoRaWAN} Stack Integration and Analysis}

The initial phase involved the deployment and analysis of the \texttt{zephyr-os-example-\gls{LoRaWAN}} project. Key tasks included:
\begin{itemize}
    \item Configuration of device identification parameters, including Device \gls{EUI}, Join \gls{EUI}, and Application Key for \gls{OTAA} activation.
    \item Integration of the \gls{LoRaWAN} stack within the Zephyr \gls{RTOS} environment.
    \item Verification of network join procedures, uplink and downlink transmission through a gateway located at the University Campus.
\end{itemize}

\subsubsection{Phase 2: Application Development and Remote Monitoring}

Building on the connectivity baseline, the plant monitoring application was integrated into the \gls{LoRaWAN} communication loop:
\begin{itemize}
    \item Implementation of data acquisition for sensors and send it to the ResIoT platform.
    \item Design of a custom LUA payload decoder for the ResIoT network server to process incoming data.
    \item Creation of a remote dashboard featuring map widgets for real-time tracking and historical data.
    \item Development of a downlink command handler to remotely change the \gls{RGB} \gls{LED} (OFF, Green or Red) from the ResIoT platform.
\end{itemize}

\subsubsection{Phase 3: Payload Optimization and Efficiency}

To maximize data throughput and adhere to the 30-byte message limit, significant optimizations were implemented:
\begin{itemize}
    \item Transition from string-based messaging to a binary-packed structure using an \texttt{\_\_attribute\_\_((packed))} C struct to reduce payload size.
    \item Conversion of geographical coordinates and sensor values to appropriate data types to minimize byte usage.
    \item Refinement of the LUA decoder to handle the optimized binary format and update the ResIoT variables accordingly.
\end{itemize}
