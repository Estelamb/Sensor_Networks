\section{Conclusions and Future Works}

\subsection{Conclusions}

This project has successfully achieved the design and implementation of a complete \gls{IoT}-based Plant Monitoring System using the STM32WL55JC microcontroller and the Zephyr \gls{RTOS}. The developed system integrates multiple analog and digital sensors to acquire environmental, physical, and positional data, demonstrating robust multitasking through a multi-threaded architecture. By leveraging dedicated threads for sensors and \gls{GPS} data, the system effectively manages concurrent data acquisition and \gls{LoRaWAN} communication.

All mandatory system requirements were fulfilled, including periodic sensor acquisition every 60 seconds and visual feedback through a \gls{RGB} \gls{LED}. A critical milestone completed was the optimization of the communication \gls{LoRaWAN} stack, transitioning to a 30-byte packed binary structure to maximize transmission efficiency. The software architecture utilized atomic variables to protect shared resources, ensuring system stability and preventing data corruption during thread context switches.

The integration with the ResIoT platform enabled a comprehensive data pipeline, from physical sensing to cloud visualization through an interactive dashboard. The implementation of \gls{OTAA} activation and a custom LUA payload decoder ensured secure connectivity and the accurate transformation of raw hexadecimal data into human-readable metrics. Furthermore, the successful deployment of a downlink command handler allows for real-time remote interaction with the node's hardware.

\subsection{Future Works}

Although the system meets all specified requirements, several improvements and extensions could be considered in future iterations to enhance performance, accuracy, and scalability:

\begin{itemize}
    \item \textbf{Expanded Sensing Capabilities:} Adding additional simple sensors, such as a CO2 probe or a pH sensor, to provide a more comprehensive view of the plant's physiological health.
    \item \textbf{Physical Protection:} Designing a weather-resistant enclosure for the STM32WL55JC and its breadboard components to allow for testing in actual outdoor garden environments.
    \item \textbf{Power Consumption Profile:} Implementing sleep modes between the 60-second transmission intervals to further extend battery life for long-term field deployment.
    \item \textbf{Adaptive Sampling:} Developing logic to adjust the transmission interval dynamically based on soil moisture levels or environmental criticalities (e.g., increasing frequency during drought conditions).
    \item \textbf{Edge Computing:} Integrating local data processing to detect leaf color anomalies or sudden orientation shifts (acceleration) locally on the \gls{MCU}, sending "Alarm" flags to reduce unnecessary network traffic.
    \item \textbf{Advanced Visualization:} Expanding the system with predictive analytics to forecast irrigation needs based on historical temperature and soil moisture trends.
\end{itemize}
