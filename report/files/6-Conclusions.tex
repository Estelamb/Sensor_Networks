\section{Conclusions and Future Works}

\subsection{Conclusions}

This project has successfully achieved the design and implementation of a complete IoT-based Plant Monitoring System using the STM32WL55JC microcontroller and the Zephyr \gls{RTOS}. The developed system integrates multiple analog and digital sensors to acquire environmental, physical, and positional data, demonstrating robust multitasking, peripheral management, and real-time operation.

All mandatory system requirements were fulfilled, including periodic sensor acquisition, multi-mode operation (Test, Normal, and Advanced modes), statistical processing, GPS integration, and visual feedback through an RGB LED. The software architecture was carefully structured into modular drivers, dedicated threads, and synchronized shared resources, ensuring scalability, maintainability, and clarity of operation.

The use of Zephyr \gls{RTOS} enabled efficient thread management, synchronization, and timing control, while also facilitating debugging, logging, and performance analysis. Extensive unit, system, and external testing confirmed the stability, correctness, and robustness of the implementation under different operational conditions.

Overall, the project demonstrates a full embedded \gls{IoT} systems development lifecycle, from requirements analysis and hardware integration to software design, validation, and documentation. The resulting platform provides a solid foundation for more advanced \gls{IoT} monitoring applications in agriculture, environmental sensing, and smart systems.

\subsection{Future Work}

Although the system meets all specified requirements, several improvements and extensions could be considered in future iterations to enhance performance, accuracy, and scalability:

\begin{itemize}
    \item \textbf{Hardware-based \gls{PWM} generation}: The current implementation emulates \gls{PWM} behavior in software for RGB LED intensity control. A natural improvement would be to use dedicated hardware timers to generate \gls{PWM} signals. This would reduce CPU load, improve timing accuracy, and provide smoother and more precise brightness control.

    \item \textbf{Improved thread synchronization mechanisms}: The system currently relies on semaphores for thread synchronization. Future versions could explore the use of event-based synchronization primitives (such as Zephyr events or message queues) to achieve more control, improved determinism, and clearer synchronization semantics between threads.

    \item \textbf{Power optimization}: Additional low-power techniques could be implemented, such as deeper sleep states, dynamic sensor activation, and adaptive sampling rates based on environmental conditions.

    \item \textbf{Wireless communication integration}: Leveraging the STM32WL55JC's integrated sub-GHz radio to transmit sensor data wirelessly (e.g., using \gls{LoRa} or other \gls{LPWAN} protocols) would significantly enhance the system's applicability in real-world deployments.

    \item \textbf{Data logging and cloud integration}: Persistent storage of measurements and integration with cloud platforms or dashboards would enable long-term analysis, remote monitoring, and predictive analytics.

    \item \textbf{Advanced data processing}: The addition of anomaly detection, trend analysis, or machine learning techniques could further improve plant health assessment and decision-making capabilities.
\end{itemize}

These potential enhancements would build upon the solid architecture developed in this project, transforming the system into a more powerful, efficient, and scalable \gls{IoT} monitoring solution.