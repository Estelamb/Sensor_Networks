\section{Overview and Introduction}

\subsection{Document Overview}

This document defines the technical specifications, development requirements, hardware architecture, software components, and operational modes of the \gls{IoT}-based Plant Monitoring System to be implemented during the course \textit{Embedded Platforms and Communications for IoT}. The objective of this specification is to establish a clear and comprehensive reference for the design, implementation, verification, and assessment of the final embedded system.

\subsection{Project Introduction}

The purpose of the final project is to design and implement a fully functional \gls{IoT} platform capable of monitoring the environmental conditions and physiological state of a plant throughout its lifecycle. Such monitoring is essential for applications such as greenhouse automation, precision agriculture, plant health diagnostics, and traceability in plant transportation and storage.

The system must continuously acquire, process, and report multiple physical variables, including temperature, relative humidity, ambient light intensity, soil moisture, and colour characteristics of a plant leaf. Additional inertial data are collected through an accelerometer to detect events such as impacts, falls, tilting, or abnormal movements. The global position of the plant is obtained via a \gls{GPS} module, ensuring timestamped logging of all monitored parameters.

This project integrates both hardware and software development activities. Students must interface several digital and analog sensors, configure low-level peripherals (\gls{I2C}, \gls{UART}, \gls{ADC}, \gls{GPIO}, \gls{PWM}), and implement multitasking using Zephyr \gls{RTOS}. 

The final embedded application must operate in distinct modes, manage periodic measurements, compute statistical parameters, handle event-based notifications, and provide visual feedback via a \gls{RGB} \gls{LED}.

\subsection{Summary of the Work Done}

This section provides a consolidated overview of the work completed throughout the development of the \gls{IoT}-based Plant Monitoring System. The tasks performed encompass the full engineering workflow, including requirement analysis, hardware integration, software design, system implementation, testing, and verification.

\subsubsection{Requirements Analysis}

The project began with an in-depth study of the provided technical specifications and sensor documentation. All functional, hardware, and timing requirements were reviewed to establish a clear design baseline. This included understanding the sensing ranges, communication interfaces, and operating constraints imposed by the STM32WL55JC microcontroller and Zephyr \gls{RTOS}.

\subsubsection{Hardware Integration}

The hardware development stage consisted of identifying, wiring, and validating all sensor interfaces:

\begin{itemize}
    \item STM32WL55JC microcontroller as the central processing unit.
    \item Integration of the Si7021 temperature and humidity sensor using the \gls{I2C} bus.
    \item Connection and calibration of the HW5P-1 phototransistor for ambient light measurement.
    \item Analog acquisition and scaling of the SEN-13322 soil moisture probe.
    \item Digital configuration of the TCS34725 colour sensor over \gls{I2C}.
    \item Setup of the MMA8451Q accelerometer for multi-axis measurements.
    \item Interfacing and configuring the Adafruit \gls{GPS} module via \gls{UART}.
    \item Implementation of a \gls{RGB} \gls{LED} driver using \gls{PWM} emulation for status indication.
\end{itemize}

\subsubsection{Software Development}

Software implementation was carried out using Zephyr \gls{RTOS}, structured with a multitasking architecture. Key activities included:

\begin{itemize}
    \item Configuration of device tree overlays and config options for all peripherals.
    \item Development of sensor drivers and low-level routines for \gls{ADC}, \gls{I2C}, \gls{UART}, and \gls{GPIO}.
    \item Design of individual threads for periodic sampling, data processing, \gls{GPS} acquisition, and mode management.
    \item Implementation of Test Mode, Normal Mode, and optional Advanced Mode according to system requirements.
    \item Integration of statistical processing to compute hourly mean, minimum, and maximum values.
    \item Implementation of colour-based alert mechanisms based on out-of-range sensor values.
\end{itemize}

Zephyr's logging and shell utilities were used extensively for debugging and validation.

\subsubsection{System Testing and Validation}

The complete system was evaluated across all operational modes:

\begin{itemize}
    \item Verification of measurement accuracy and stability under Test Mode.
    \item Long-term monitoring and statistical computation under Normal Mode.
    \item Correct operation of the mode-switching mechanism through the push button.
    \item Validation of \gls{RGB} \gls{LED} behaviour for both colour detection and alert signalling.
    \item \gls{GPS} time synchronization and conversion to local time for timestamp generation.
    \item Stress testing of the application to identify stack usage limits and race conditions.
\end{itemize}

All mandatory functionalities were confirmed to meet the specifications, with optional enhancements explored where possible.

\subsubsection{Final Deliverables}

The completed work includes:

\begin{itemize}
    \item A functional embedded system integrating all sensors and the STM32WL55JC \gls{MCU}.
    \item Clean, documented source code developed under Zephyr \gls{RTOS}.
    \item A complete technical report detailing the system design, implementation, and results.
    \item Final project documentation and code submitted according to course requirements.
\end{itemize}

Overall, the work conducted demonstrates a full-cycle embedded systems development process, covering hardware, software, real-time processing, testing, and documentation.
