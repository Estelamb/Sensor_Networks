\section{Overview and Introduction} \todo{revisar}

\subsection{Document Overview}

This document defines the technical specifications, development requirements, hardware architecture, software components, and operational modes of the \gls{IoT}-based Plant Monitoring System implemented during the course \textit{Sensor Networks}. The objective of this specification is to establish a clear and comprehensive reference for the design, implementation, and verification of a system that integrates long-range connectivity via \gls{LoRaWAN}.

\subsection{Project Introduction}

The main goal of Project 1 is to enable \gls{LoRaWAN} connectivity on an STM32-based \gls{IoT} node running the Zephyr \gls{RTOS}. The system is designed to monitor the environmental conditions and physiological state of a plant throughout its life. This capability is crucial for applications such as high-value species monitoring, greenhouse instrumentation, and logistics traceability where environmental history affects the final market price.

The system utilizes the on-board radio transceiver of the STM32WL55JC platform to transmit data to a network server. It continuously acquires variables including temperature, humidity, ambient light, soil moisture, and leaf colour. Additionally, a 3-axis accelerometer detects physical impacts or falls, while an external \gls{GPS} module provides precise location data. All parameters are reported periodically through \gls{LoRaWAN} uplinks to the ResIOT \gls{IoT} Platform.

\subsection{Summary of the Work Done}

This section provides a consolidated overview of the work completed, which builds upon the sensor management hardware and software developed in the previous subject. The current work focuses on the integration of the \gls{LoRaWAN} stack and cloud-based data visualization.

\subsubsection{\gls{LoRaWAN} Stack Integration and Analysis}

The initial phase involved the deployment and analysis of the \texttt{zephyr-os-example-\gls{LoRaWAN}} project. Key tasks included:
\begin{itemize}
    \item Configuration of device identification parameters, including Device \gls{EUI}, Join \gls{EUI}, and Application Key for \gls{OTAA} activation.
    \item Integration of the \gls{LoRaWAN} stack within the Zephyr \gls{RTOS} environment.
    \item Verification of network join procedures and uplink transmission through a gateway located at the University Campus.
\end{itemize}

\subsubsection{Application Development and Remote Monitoring}

Building on the connectivity baseline, the plant monitoring application was integrated into the \gls{LoRaWAN} communication loop:
\begin{itemize}
    \item Implementation of data acquisition routines for environmental sensors and \gls{GPS} \gls{NMEA} frame parsing.
    \item Design of a custom LUA payload decoder for the ResIOT network server to process incoming hexadecimal data.
    \item Creation of a remote dashboard featuring map widgets for real-time tracking and historical data charts for sensor values.
    \item Development of a downlink command handler to remotely change the \gls{RGB} \gls{LED} colour (OFF, Green, Red) from the ResIOT platform.
\end{itemize}

\subsubsection{Payload Optimization and Efficiency (Phase 3)}

To maximize data throughput and adhere to the 30-byte message limit, significant optimizations were implemented:
\begin{itemize}
    \item Transition from string-based messaging to a binary-packed structure using an \texttt{\_\_attribute\_\_((packed))} C struct to reduce payload size.
    \item Conversion of geographical coordinates and sensor values to appropriate data types to minimize byte usage.
    \item Refinement of the LUA decoder to handle the optimized binary format and update the ResIOT variables accordingly.
\end{itemize}

\subsubsection{System Validation}

The complete system was validated through functional assessments:
\begin{itemize}
    \item Confirmation of successful \gls{OTAA} joins and stable data transmission intervals.
    \item Verification of end-to-end data integrity from the physical sensors to the mobile application.
    \item Testing of the downlink control path for real-time actuator interaction.
\end{itemize}

\subsubsection{Final Deliverables}

The completed work includes the optimized source code for the STM32WL55JC node, the LUA decoding logic for the network server, and this technical report detailing the implementation of a long-range \gls{IoT} monitoring solution.