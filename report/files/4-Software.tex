\section{Software Organization}

\subsection{Description of the Global Software Architecture}

The Plant Monitoring System is a multi-threaded embedded application developed using the Zephyr \gls{RTOS}. It integrates multiple sensors and peripherals, manages several execution threads, uses atomic shared structures for inter-thread communication, and supports three distinct operating modes that govern its behaviour.

This section provides a unified description of system operation, the behaviour of each mode, the internal synchronization mechanisms, and the shared peripheral configuration structures.

\subsubsection{Detailed System Behaviour per Operating Mode}

The system operates according to the value of the \texttt{system\_mode\_t} enumeration, which cycles through three states: \textbf{TEST\_MODE}, \textbf{NORMAL\_MODE}, and \textbf{ADVANCED\_MODE}. The user button triggers the transitions.

\begin{figure}[H]
    \centering
    \includegraphics[width=0.99\textwidth]{images/state_machine.png}
    \caption{System behaviour}
    \label{fig:state_machine}
\end{figure}

In \textbf{TEST\_MODE}, the system performs simple acquisition and visualisation tasks:

\begin{enumerate}
    \item The periodic measurement timer (\texttt{main\_timer}) is started with the test-mode rate (2 seconds).
    \item The sensor and \gls{GPS} threads are triggered to acquire new samples.
    \item The main thread waits for:
    \begin{itemize}
        \item \texttt{main\_sensors\_sem}: Signals completion of environmental sensor acquisition,
        \item \texttt{main\_gps\_sem}: Signals completion of \gls{GPS} sampling.
    \end{itemize}
    \item The main thread reads the shared atomic measurement structure and determines the dominant colour measured by the \gls{RGB} sensor.
    \item The \gls{RGB} \gls{LED} is set to the dominant raw channel (red, green, or blue).
    \item The measurements are displayed via the serial console.
    \item At each timer expiration, the \texttt{main\_sem} semaphore is released to repeat the mode.
    \item No limit checking, no statistics, and no warning indicators are active.
\end{enumerate}

TEST\_MODE is primarily a hardware validation mode to verify that all sensors are functional.

\begin{figure}[H]
    \centering
    \includegraphics[width=0.8\textwidth]{images/test_mode.png}
    \caption{Test mode behaviour}
    \label{fig:test_mode}
\end{figure}

In \textbf{NORMAL\_MODE}, the system executes the full environmental monitoring workflow:

\begin{enumerate}
    \item The measurement timer is started with the normal-mode sampling rate (30 seconds).
    \item Sensor and \gls{GPS} threads acquire data and raise their semaphores.
    \item The main thread retrieves all sensor values from the atomic shared structure.
    \item Limit checking is performed for temperature, humidity, moisture, brightness, and acceleration parameters.
    \item Violations raise bits of the atomic \texttt{rgb\_flags} field, allowing multiple alarms simultaneously.
    \item A dedicated \gls{RGB} warning timer is active: at each tick it reads the \texttt{rgb\_flags} and updates the \gls{RGB} \gls{LED} with colour-coded warnings.
    \item The system maintains statistical data, which is shown every hour:
    \begin{itemize}
        \item running minimum and maximum,
        \item running mean values,
        \item colour frequency counts.
    \end{itemize}
    \item At each timer expiration, the \texttt{main\_sem} semaphore is released to repeat the mode.
\end{enumerate}

NORMAL\_MODE is the complete monitoring and alert mode.

\begin{figure}[H]
    \centering
    \includegraphics[width=0.99\textwidth]{images/normal_mode.png}
    \caption{Normal mode behaviour}
    \label{fig:normal_mode}
\end{figure}

In \textbf{ADVANCED\_MODE}, the system focuses on reproducing the sensed colour with high fidelity:

\begin{enumerate}
    \item The measurement timer is started with the advanced-mode sampling rate (30 seconds).
    \item Sensor and \gls{GPS} threads acquire data and raise their semaphores.
    \item The main thread retrieves all sensor values from the atomic shared structure.
    \item Colour normalization is performed using the clear channel of the colour sensor, and shown.
    \item The \gls{RGB} \gls{LED} is set to the normalized colour values for accurate reproduction (emulated PWM).
    \item At each timer expiration, the \texttt{main\_sem} semaphore is released to repeat the mode.
    \item No limit checking, no statistics, and no warning indicators are active.
\end{enumerate}

ADVANCED\_MODE is a pure high-resolution colour rendering mode.

\begin{figure}[H]
    \centering
    \includegraphics[width=0.8\textwidth]{images/advanced_mode.png}
    \caption{Advanced mode behaviour}
    \label{fig:advanced_mode}
\end{figure}


\subsubsection{Mode Transition Mechanism}

The user button controls mode transitions. The button interrupt triggers a deferred work handler, ensuring transitions occur in thread context. The mode progression is cyclic:

\[
\texttt{TEST\_MODE} \rightarrow \texttt{NORMAL\_MODE} \rightarrow \texttt{ADVANCED\_MODE} \rightarrow \texttt{TEST\_MODE}
\]

During a mode change:

\begin{itemize}
    \item Timers associated with the previous mode are stopped,
    \item Mode-specific timers are started (measurement, \gls{RGB} warning, statistics),
    \item \glspl{LED} are updated to reflect the new mode,
    \item \texttt{main\_sem} is released to unblock the main loop.
\end{itemize}

\subsubsection{Semaphore Synchronization and Thread Interactions}

Inter-thread coordination relies on multiple named semaphores.

\begin{itemize}
    \item \textbf{Measurement Timer Semaphore (\texttt{main\_sem})}
    
    Released by \texttt{main\_timer} to:

    \begin{itemize}
        \item Wake the main thread at the configured sampling frequency to repeat the mode.
        \item Wake the main thread when a mode transition occurs (button pressed).
    \end{itemize}

    \item \textbf{Trigger Semaphores (\texttt{sensors\_sem} and \texttt{gps\_sem})}

    These semaphores are released by \texttt{main\_timer} to notify the respective threads to begin their measurement cycles.

    \item \textbf{Sensor Data Semaphore (\texttt{main\_sensors\_sem})}

    Raised by the sensor acquisition thread after reading:

    \begin{itemize}
        \item Temperature and humidity.
        \item \gls{RGB} colour channels.
        \item Brightness.
        \item Soil moisture.
        \item Accelerometer values.
    \end{itemize}

    The main thread blocks until this semaphore is obtained, ensuring consistent data.

    \item \textbf{\gls{GPS} Data Semaphore (\texttt{main\_gps\_sem})}

    Raised by the \gls{GPS} thread when new \gls{GPS} information (latitude, longitude, altitude, satellites, timestamp) is available.

    The main thread blocks until this semaphore is obtained, ensuring consistent data.
\end{itemize}

Together, these semaphores serialize measurement flow and guarantee no partial or inconsistent samples.

\subsubsection{Atomic Shared Measurement Structure}

Sensor data is stored in a global structure using only atomic variables:

\begin{lstlisting}[language=C, caption={Sensor data structure with atomic fields}]
struct system_measurement {
    atomic_t brightness;  /**< Latest ambient brightness (0-100%). */
    atomic_t moisture;    /**< Latest soil moisture (0-100%). */

    atomic_t accel_x_g;   /**< Latest X-axis acceleration (in g). */
    atomic_t accel_y_g;   /**< Latest Y-axis acceleration (in g). */
    atomic_t accel_z_g;   /**< Latest Z-axis acceleration (in g). */

    atomic_t temp;        /**< Latest temperature (C). */
    atomic_t hum;         /**< Latest relative humidity (%RH). */

    atomic_t red;         /**< Latest red color value (raw). */
    atomic_t green;       /**< Latest green color value (raw). */
    atomic_t blue;        /**< Latest blue color value (raw). */
    atomic_t clear;       /**< Latest clear color channel value (raw). */

    atomic_t gps_lat;     /**< Latest GPS latitude (degrees). */
    atomic_t gps_lon;     /**< Latest GPS longitude (degrees). */
    atomic_t gps_alt;     /**< Latest GPS altitude (meters). */
    atomic_t gps_sats;    /**< Latest number of satellites in view. */
    atomic_t gps_time;    /**< Latest GPS timestamp (float or encoded). */
};
\end{lstlisting}

Properties:

\begin{itemize}
    \item Lock-free thread-safe communication.
    \item Each measurement updated independently.
    \item the main thread reads all values without risk of torn writes.
\end{itemize}

\subsubsection{Peripheral Configuration Structure}

All peripherals and synchronization objects are referenced through a single shared structure:

\begin{lstlisting}[language=C, caption={Peripheral configuration and synchronization structure}]
struct system_context {
    struct adc_config *phototransistor; /**< Phototransistor ADC configuration. */
    struct adc_config *soil_moisture;   /**< Soil moisture ADC configuration. */

    struct i2c_dt_spec *accelerometer;  /**< Accelerometer I2C device specification. */
    uint8_t accel_range;                /**< Accelerometer full-scale range (e.g., 2G, 4G, 8G). */

    struct i2c_dt_spec *temp_hum;       /**< Temperature and humidity sensor I2C specification. */
    struct i2c_dt_spec *color;          /**< Color sensor I2C device specification. */
    struct gps_config *gps;             /**< GPS module configuration. */

    struct k_sem *main_sensors_sem;     /**< Semaphore for main-to-sensors synchronization. */
    struct k_sem *main_gps_sem;         /**< Semaphore for main-to-GPS synchronization. */
    struct k_sem *sensors_sem;          /**< Semaphore to trigger sensor measurement. */
    struct k_sem *gps_sem;              /**< Semaphore to trigger GPS measurement. */
};
\end{lstlisting}

In this way, the peripherals are configured once at startup and passed to all threads and modules that require access.

\subsubsection{Secure Initialization}

In case of initialization failures (e.g., \gls{I2C} device not found), the system doesn't execute the program:

\begin{lstlisting}[language=C, caption={Secure initialization code}]
    /* Initialize peripherals */
    if (gps_init(&gps)) {
        printk("GPS initialization failed - Program stopped\n");
        return -1;
    }
    if (adc_init(&pt)) {
        printk("Phototransistor initialization failed - Program stopped\n");
        return -1;
    }
    if (adc_init(&sm)) {
        printk("Soil moisture sensor initialization failed - Program stopped\n");
        return -1;
    }
    if (accel_init(&accel, ACCEL_RANGE)) {
        printk("Accelerometer initialization failed - Program stopped\n");
        return -1;
    }
    if (temp_hum_init(&th, TEMP_HUM_RESOLUTION)) {
        printk("Temperature/Humidity sensor initialization failed - Program stopped\n");
        return -1;
    }
    if (color_init(&color, COLOR_GAIN, COLOR_INTEGRATION_TIME)) {
        printk("Color sensor initialization failed - Program stopped\n");
        return -1;
    }
    if (led_init(&leds) || led_off(&leds)) {
        printk("LED initialization failed - Program stopped\n");
        return -1;
    }
    if (rgb_led_init(&rgb_leds) || rgb_led_off(&rgb_leds)) {
        printk("RGB LED initialization failed - Program stopped\n");
        return -1;
    }
    if (button_init(&button))  {
        printk("Button initialization failed - Program stopped\n");
        return -1;
    }
    if (button_set_callback(&button, button_isr)) {
        printk("Button callback setup failed - Program stopped\n");
        return -1;
    }

\end{lstlisting} 

\autoref{fig:no_init} shows an example of the log printed when initialization fails. In this case, the accelerometer is disconnected.

\begin{figure}[H]
    \centering
    \includegraphics[width=0.7\textwidth]{images/no_init.png}
    \caption{Stopped initialisation}
    \label{fig:no_init}
\end{figure}

\subsubsection{Device Disconnected}

In case one device is disconnected during operation, an error message is printed, but the program continues running. It shows the last valid measurement for that sensor. If the device is reconnected, normal operation resumes.

\begin{itemize}
    \item In GREEN, the log that shows the sensor is disconnected.
    \item In BLUE is highlighted the same value of the measurement, demonstrating that the system shows the last valid measurement. 
    \item In RED, there isn't log of sensor disconnected, because the sensor is connected again. The value of the measurement is updated.
\end{itemize}

\begin{figure}[H] 
    \centering
    \includegraphics[width=0.9\textwidth]{images/no_measurements.png}
    \caption{Disconnected and Connected sensor measurements}
    \label{fig:no_measure}
\end{figure}

\clearpage
\subsection{Modules}

The program is structured into several self-contained modules, each responsible for a specific hardware interface or functionality.

\begin{figure}[H]
    \centering
    \includegraphics[width=0.55\textwidth]{images/modules.png}
    \caption{Modules overview}
    \label{fig:modules}
\end{figure}

The main modules are used by the main program and the threads to interact with sensors and peripherals. This is done by adding them as source files and include directories in the \texttt{CMakeLists.txt} file, as shown below:

\begin{lstlisting}[caption={CMakeLists.txt module integration}]
    # SPDX-License-Identifier: Apache-2.0

    cmake_minimum_required(VERSION 3.20.0)

    find_package(Zephyr REQUIRED HINTS $ENV{ZEPHYR_BASE})
    project(plant_monitoring_system)

    target_sources(app PRIVATE 
        src/main.c
        src/sensors_thread.c
        src/gps_thread.c
        src/sensors/led/rgb_led.c
        src/sensors/led/board_led.c
        src/sensors/adc/adc.c
        src/sensors/user_button/user_button.c
        src/sensors/i2c/i2c.c
        src/sensors/i2c/accel.c
        src/sensors/i2c/color.c
        src/sensors/i2c/temp_hum.c
        src/sensors/gps/gps.c
    )

    target_include_directories(app PRIVATE
        src/sensors/led
        src/sensors/adc
        src/sensors/user_button
        src/sensors/i2c
        src/sensors/gps
    )
\end{lstlisting}

\subsubsection{adc.c and adc.h}

The \texttt{adc.c} and \texttt{adc.h} modules provide a hardware abstraction layer for reading analogue values from the phototransistor and soil-moisture sensors using the Zephyr \gls{ADC} \gls{API}. Their purpose is to encapsulate \gls{ADC} initialisation, configuration, and acquisition into a self-contained interface that the sensors thread can use without exposing low-level driver details.

This library is responsible for preparing the \gls{ADC} peripheral, configuring its channels, and performing synchronous conversions on demand. It ensures that all \gls{ADC} reads comply with a consistent configuration (resolution, reference, acquisition time, and oversampling) and that raw sample values are returned in a unified format to the rest of the system.

\begin{itemize}
    \item \textbf{Peripheral Initialisation}: Loads the \gls{ADC} device from the device tree and configures the hardware according to project requirements. This includes selecting resolution, reference voltage, acquisition time, and optional oversampling.

    \item \textbf{Channel Setup}: Each analogue sensor has an associated \gls{ADC} channel configured through a dedicated structure. The module ensures correct pin routing and channel mapping according to the board overlay.

    \item \textbf{Synchronous ADC Sampling}: Provides a blocking \gls{API} that triggers a single conversion and returns the measured sample. This prevents concurrency issues by guaranteeing that read operations finish before returning control.

    \item \textbf{Unified Abstraction for Higher-Level Modules}: The sensors thread and the system context only interact with a clean, high-level interface without needing to manage \gls{ADC} device handles, channels, or Zephyr-specific configuration fields.

    \item \textbf{Validation and Error Handling}: Detects device-not-found conditions, invalid configurations, or read failures, forwarding errors to the main system so that appropriate recovery or safe behaviour can occur.

    \item \textbf{Scalability for Additional Channels}: The design allows new analogue sensors to be added by defining a new channel configuration and calling the same acquisition \gls{API}, without modifying existing code.
    
    \item \textbf{Separation of Configuration from Logic}: The module centralises all \gls{ADC} configuration parameters in one place, ensuring future modifications (e.g., resolution, gain, sampling frequency) do not propagate across the project.
\end{itemize}

This \gls{ADC} library is intended to be reused by any component that requires analogue-to-digital conversions while maintaining a clean separation between hardware-specific details and system-level functionality.

\subsubsection{gps.c and gps.h}

The \texttt{gps.c} and \texttt{gps.h} modules implement a GPS interface based on \gls{UART}-driven reception of \gls{NMEA} sentences. Their purpose is to offer a simple, self-contained parser for \gls{GGA} (or \gls{GNGGA}) frames and provide the rest of the system with clean, validated geographic data without exposing \gls{UART} or interrupt logic.

The library configures the \gls{UART} peripheral, enables interrupt-driven reception, reconstructs \gls{NMEA} lines in the background, and parses the relevant fields when a complete \gls{GGA} sentence is detected. Once valid GPS information is available, the module updates an internal \texttt{gps\_data\_t} structure and releases a semaphore so that higher-level threads can safely retrieve the most recent fix.

\begin{itemize}
    \item \textbf{\gls{UART}-Based \gls{GPS} Initialisation}: The module validates the configuration, checks device readiness, attaches the \gls{ISR}, and enables \gls{RX} interrupts. This ensures autonomous background reception of \gls{NMEA} data.

    \item \textbf{Interrupt-Driven \gls{NMEA} Line Reconstruction}: Bytes received from the \gls{UART} \gls{FIFO} are accumulated into an internal buffer until a newline character is found.

    \item \textbf{\gls{GGA} Sentence Parsing}: Extracts latitude, longitude, altitude, \gls{HDOP}, satellite count, and \gls{UTC} time from a standard \gls{GGA} frame. The parser handles missing or malformed fields gracefully and only accepts complete, coherent entries.

    \item \textbf{\gls{NMEA}-to-Degrees Conversion}: Converts coordinates from \gls{NMEA} format (DDMM.MMMM or DDDMM.MMMM) to decimal degrees, applying hemisphere correction. This provides the system with immediately usable geographic values.

    \item \textbf{Thread Synchronisation via Semaphore}: When new valid data is parsed, a semaphore is released so that the \gls{GPS} thread or main thread can block until a fresh fix is available. This avoids polling and reduces \gls{CPU} usage.

    \item \textbf{Internal Data Buffering}: The module maintains an internal instance of \texttt{gps\_data\_t} storing the last valid parsed frame. Consumers obtain a copy, ensuring thread safety without exposing shared mutable structures.

    \item \textbf{Timeout-Aware Data Retrieval}: The high-level \gls{API} allows callers to wait indefinitely, for a fixed period, or return immediately if no new \gls{GGA} sentence has been received.
\end{itemize}

This \gls{GPS} interface provides a robust and maintainable foundation for acquiring geographic data in real time, isolating UART management and parsing details from the rest of the application.

\subsubsection{i2c.c and i2c.h}

The \texttt{i2c.c} and \texttt{i2c.h} modules implement a small set of helper functions designed to simplify register-level communication with \gls{I2C} devices in Zephyr. Their purpose is to provide a clean, reusable interface for reading and writing device registers using only a devicetree \texttt{i2c\_dt\_spec}, avoiding repetitive low-level code in sensor drivers.

The library encapsulates common \gls{I2C} access patterns into concise functions. These utilities internally rely on Zephyr's \texttt{i2c\_write\_read\_dt}, \texttt{i2c\_write\_dt}, and \texttt{i2c\_is\_ready\_dt} \glspl{API}, ensuring compatibility with any \gls{I2C} peripheral described in the system devicetree.

Because many sensors require register-based configuration and multi-byte reads, this module centralises these operations and presents a uniform interface that higher-level modules can reuse safely.

\begin{itemize}
    \item \textbf{Multi-Register Read Helper}: Reads an arbitrary number of consecutive registers starting at a given address, abstracting the common write-then-read transaction pattern.

    \item \textbf{Single-Register Write Helper}: Writes one byte to a specified register, a frequent requirement for sensor configuration and control registers.

    \item \textbf{Device Readiness Check}: Verifies that the \gls{I2C} device is present, powered, and ready before attempting communication. Provides clear error reporting if the device is not reachable.

    \item \textbf{Consistent Devicetree-Based Access}: All functions operate on \texttt{i2c\_dt\_spec} descriptors, ensuring that pin routing, bus selection, addressing and timing come directly from the devicetree.

    \item \textbf{Reusability for Multiple Sensor Drivers}: Higher-level modules (accelerometer, colour sensor, temperature/humidity sensor, etc.) use these helpers to avoid code duplication and maintain consistency across all \gls{I2C} devices.
    
    \item \textbf{Error Propagation}: Returns standard negative errno codes, allowing calling modules to handle failures predictably and implement fallback or retry mechanisms.
\end{itemize}

This \gls{I2C} helper library provides a clean and robust foundation for register-based communication with any \gls{I2C} sensor or peripheral in the system.

\subsubsection{accel.c and accel.h}

The \texttt{accel.c} and \texttt{accel.h} modules implement a 3-axis accelerometer driver over \gls{I2C}, providing initialization, configuration, raw data acquisition, and unit conversion utilities. The module abstracts all register-level interaction and exposes a clean interface for obtaining acceleration data in either raw counts, g units, or m/s².

During initialization, the library verifies device identity via the \texttt{WHO\_AM\_I} register, transitions the sensor into standby mode, configures its measurement range, and finally activates continuous measurement mode. The project uses the ±2g range, which maximizes sensitivity for environmental and motion-tracking applications.

Raw acceleration values for X, Y, and Z axes are obtained through a single burst read of six consecutive registers. This ensures atomic acquisition of all three axes and prevents axis desynchronization. Each pair of bytes contains a 14-bit left-aligned signed measurement, which the library re-aligns before returning to higher-level modules.

All \gls{I2C} communication relies on the generic register helpers defined in the \gls{I2C} module, keeping the driver compact and uniform with the rest of the system.

\begin{itemize}
    \item \textbf{Device Initialization and Identity Check}: Reads the \texttt{WHO\_AM\_I} register to validate the sensor's presence before configuration or data acquisition.

    \item \textbf{Standby and Active Mode Control}: Ensures that configuration registers are only modified while the device is in standby mode, as required by the hardware design. The module automatically returns the device to active mode after configuration.

    \item \textbf{Measurement Range Configuration}: Supports ±2g, ±4g, and ±8g ranges via the \texttt{XYZ\_DATA\_CFG} register. The system uses ±2g for improved resolution and noise performance. 

    \item \textbf{Burst Read of 6 Output Registers}: Performs a single multi-register transaction to retrieve X, Y, and Z values consistently, minimizing communication overhead and avoiding partial updates.

    \item \textbf{Raw Data Alignment and Extraction}: Converts the sensor's 14-bit left-aligned format into signed 14-bit integers usable by higher-level modules.

    \item \textbf{Conversion to g Units}: Applies sensitivity scaling based on the configured range, providing a convenient floating-point representation.

    \item \textbf{Conversion to m/s²}: Converts from g units to SI units using standard gravity, enabling direct use in physical calculations or movement detection algorithms.
\end{itemize}

This accelerometer module provides a robust basis for motion sensing, offering clean abstractions for configuration, raw acquisition, and physical-unit conversion while preserving full compatibility with the system's \gls{I2C} infrastructure.

\subsubsection{color.c and color.h}

The \texttt{color.c} and \texttt{color.h} modules implement a driver for the TCS34725 RGB colour sensor using the Zephyr \gls{I2C} \gls{API}. Their purpose is to provide a clean, high-level interface for configuring the device, and acquiring raw Clear/Red/Green/Blue measurements without exposing low-level register logic to the rest of the system.

During initialization, the module validates \gls{I2C} bus readiness, powers on the device, enables the internal \gls{ADC}, and applies the user-specified gain and integration time. Raw colour values are retrieved through a single auto-increment burst read of the sensor's \gls{RGBC} output registers, ensuring consistent sampling of all channels.

\begin{itemize}
    \item \textbf{\gls{I2C}-Based Sensor Initialisation}: The module verifies that the \gls{I2C} bus is ready, sends the power-on command, enables the \gls{RGBC} \gls{ADC}, and configures the device using register writes to \texttt{ATIME} and \texttt{CONTROL}. Any communication error propagates as a negative \texttt{errno} code.

    \item \textbf{Gain Configuration}: The driver supports all hardware gain settings (1$\times$, 4$\times$, 16$\times$, 60$\times$). Gain determines the analogue pre-amplification applied to each colour photodiode. Lower gains avoid saturation in bright environments, while higher gains improve low-light sensitivity.

    \item \textbf{Integration Time Configuration}: Integration time controls the duration of the \gls{ADC} light accumulation interval. Short times (e.g. 2.4ms) allow fast updates but with lower resolution, while long times (up to 700ms) significantly improve sensitivity. The selected timing constant is written directly to the \texttt{ATIME} register.

    \item \textbf{Burst Read of \gls{RGBC} Channels}: The function \texttt{color\_read\_rgb()} performs a single multi-register transaction starting at \texttt{COLOR\_CLEAR\_L}, retrieving all Clear, Red, Green, and Blue low/high bytes using auto-increment addressing. This guarantees coherence between channels and minimizes \gls{I2C} overhead.

    \item \textbf{16-bit Data Reconstruction}: Raw samples are assembled from consecutive low/high bytes and stored into a \texttt{ColorSensorData} structure. The Clear channel is reported alongside \gls{RGB} values, enabling normalisation or illumination-compensation algorithms at higher layers.

    \item \textbf{Error Handling and Consistent \gls{API}}: All functions return standard negative \texttt{errno} values on failure, ensuring predictable error reporting and uniform behaviour across all \gls{I2C}-based sensor modules.
\end{itemize}

This colour-sensor driver offers a compact and maintainable interface for acquiring raw \gls{RGBC} data, abstracting all register and timing details while remaining fully consistent with the system's \gls{I2C} infrastructure.

\subsubsection{temp\_hum.c and temp\_hum.h} 

The \texttt{temp\_hum.c} and \texttt{temp\_hum.h} modules implement a driver for the Si7021 temperature and humidity sensor over \gls{I2C}. Their purpose is to provide a synchronous, resolution-configurable interface for acquiring relative humidity (\%\gls{RH}) and temperature (C) values while abstracting all low-level \gls{I2C} communication.

The library initializes the sensor, performs a soft reset, sets the desired measurement resolution, and provides functions for reading both temperature and humidity using \textbf{Hold Master mode}. In this mode, the sensor holds the \gls{SCL} line low while a measurement is in progress, ensuring that the master waits synchronously until the data is ready.

\begin{itemize}
    \item \textbf{Device Initialization and Soft Reset}: \texttt{temp\_hum\_init()} validates the \gls{I2C} bus, issues a soft reset (\texttt{TH\_RESET}), waits for the sensor to stabilize, and writes the resolution to User Register 1.

    \item \textbf{Hold Master Mode Measurements}: Both \texttt{temp\_hum\_read\_humidity()} and \texttt{temp\_hum\_read\_} \texttt{temperature()} use Hold Master mode commands (\texttt{TH\_MEAS\_RH\_HOLD} and \texttt{TH\_MEAS\_TEMP\_HOLD}) to ensure synchronous reading without the need for polling or manual delays.

    \item \textbf{Relative Humidity Conversion}: Converts the 16-bit raw measurement from the sensor into \%\gls{RH} according to the Si7021 datasheet formula, clamping values to the physical range of 0-100\%.

    \item \textbf{Temperature Conversion}: Converts the 16-bit raw measurement into degrees Celsius using the datasheet formula, providing accurate environmental temperature readings.

    \item \textbf{Error Handling}: All functions return standard negative \texttt{errno} codes in case of communication or configuration failures, allowing higher-level modules to react accordingly.
\end{itemize}

This \gls{I2C}-based temperature and humidity driver offers a simple, reliable, and synchronous interface, isolating low-level communication and Hold Master timing from the rest of the application.

\subsubsection{rgb\_led.c and rgb\_led.h}

The \texttt{rgb\_led.c} and \texttt{rgb\_led.h} modules implement control of a \gls{RGB} \gls{LED} connected via three \gls{GPIO} pins (Red, Green, Blue). This module provides initialization, individual color control, and mixed color combinations through a bitmask-based bus interface. It is important to note that the \gls{RGB} \gls{LED} channels are \textbf{active-low}, meaning a logical 0 turns the \gls{LED} on, and a logical 1 turns it off.

The library initializes all \gls{GPIO} pins, verifies device readiness, and sets them to an inactive state by default. Functions allow activation of standard colors, full white, or complete off states.  

\begin{itemize}
    \item \textbf{Bus-Based \gls{GPIO} Initialization}: \texttt{rgb\_led\_init()} iterates over all pins defined in the \texttt{bus\_rgb\_led} structure, ensures the associated \gls{GPIO} device is ready, and configures the pins as outputs with an initial inactive state.

    \item \textbf{Bitmask-Controlled Color Output}: \texttt{rgb\_led\_write()} maps each bit of a 3-bit value to a corresponding \gls{LED} channel (bit0=Red, bit1=Green, bit2=Blue). This bus-like approach allows simultaneous activation of multiple channels to produce mixed colors.

    \item \textbf{Convenience Color Functions}: Functions such as \texttt{rgb\_red()}, \texttt{rgb\_green()}, \texttt{rgb\_blue()}, \texttt{rgb\_yellow()}, \texttt{rgb\_cyan()}, \texttt{rgb\_purple()}, \texttt{rgb\_white()}, and \texttt{rgb\_black()} call \texttt{rgb\_led\_write()} internally with predefined bitmasks, providing simple color selection.

    \item \textbf{Active-Low Behavior}: Since the \gls{RGB} \gls{LED} channels are active-low, a logical 0 on a \gls{GPIO} pin activates the \gls{LED}, while a logical 1 turns it off. All helper functions and the bus interface respect this behavior.

    \item \textbf{Error Handling}: Initialisation and write operations check for \gls{GPIO} device readiness and return standard negative \texttt{errno} codes on failure, allowing safe integration with higher-level modules.
\end{itemize}

This \gls{RGB} \gls{LED} module provides a reliable, bus-oriented, and reusable interface for color control, abstracting low-level active-low pin management while supporting individual colors and combined outputs.

\subsubsection{board\_led.c and board\_led.h}

The \texttt{board\_led.c} and \texttt{board\_led.h} modules implement an abstraction for controlling board \glspl{LED} using \gls{GPIO} pins. Their purpose is to provide a clean and reusable interface for turning board \glspl{LED} on/off, setting specific colors, and combining color channels via bitmask control.

The library initializes the \gls{GPIO} pins, verifies device readiness, configures outputs, and exposes functions to control individual colors or common combinations. Bitmask-based operations allow multiple channels to be activated simultaneously, producing standard colors (e.g., yellow, cyan, magenta, white).

\begin{itemize}
    \item \textbf{\gls{GPIO} Initialization}: \texttt{led\_init()} iterates over all configured pins in the \texttt{bus\_led} structure, checks device readiness, and configures each pin as an output with a default off state. Errors are reported if a pin or device is not ready.

    \item \textbf{Bitmask-Based Color Control}: \texttt{led\_write()} maps each bit of the input value to a corresponding \gls{LED} channel (bit0=Red, bit1=Green, bit2=Blue). This allows direct control of \gls{RGB} combinations in a single function call.

    \item \textbf{Convenience Functions for Colors}: The module provides higher-level functions (\texttt{red()}, \texttt{green()}, \texttt{blue()}, \texttt{red\_green()}, \texttt{green\_blue()}, \texttt{red\_blue()}, \texttt{led\_on()}, \texttt{led\_off()}) that internally call \texttt{led\_} \texttt{write()} with predefined bitmask values to simplify usage.

    \item \textbf{Error Handling}: All \gls{GPIO} operations check for errors and return standard negative \texttt{errno} codes if configuration or write fails, enabling higher-level modules to respond appropriately.
\end{itemize}

This \gls{GPIO}-based \gls{LED} driver provides a simple, reliable, and reusable interface for \gls{LED} control, isolating low-level pin management while enabling flexible color and combination handling.

\subsubsection{user\_button.c and user\_button.h}

The \texttt{user\_button.c} and \texttt{user\_button.h} modules implement a \gls{GPIO}-based user button interface with interrupt support. This driver allows initialization of a button input pin, configuration of edge-triggered interrupts, and registration of a callback function to handle button press and release events. It is designed for use with a single \gls{GPIO} pin per button, using pull-up configuration and detecting both rising and falling edges.

The library provides a clean abstraction for integrating physical buttons into the system without exposing low-level \gls{GPIO} interrupt setup details.

\begin{itemize}
    \item \textbf{\gls{GPIO}-Based Button Initialization}: \texttt{button\_init()} verifies the \gls{GPIO} device is ready, configures the pin as input with pull-up, and enables interrupts on both edges to detect presses and releases.

    \item \textbf{Edge-Triggered Interrupts}: Both rising and falling edges are detected, allowing the application to respond to button presses and releases independently.

    \item \textbf{Callback Registration}: \texttt{button\_set\_callback()} allows the application to attach an \gls{ISR} handler that will be executed in interrupt context when the configured edge is detected.

    \item \textbf{Safe Error Handling}: All functions return standard negative \texttt{errno} codes on failure (e.g., \gls{GPIO} device not ready, invalid configuration), allowing the calling module to handle errors predictably.

    \item \textbf{Lightweight \gls{ISR} Support}: The module leaves the implementation of press/release logic to the application via the registered callback, ensuring \gls{ISR} code remains minimal and safe.

    \item \textbf{Integration with Zephyr \gls{GPIO} \gls{API}}: Uses \texttt{gpio\_pin\_configure\_dt()}, \texttt{gpio\_pin\_interrupt\_} \texttt{configure\_dt()}, and \texttt{gpio\_add\_callback()} internally, abstracting Zephyr-specific details from higher-level code.
\end{itemize}

This user button module provides a robust and reusable interface for integrating physical buttons with interrupt-driven event handling, isolating low-level \gls{GPIO} configuration and edge detection logic from the application.
\clearpage
\subsection{Threads}

\subsubsection{Sensors Thread}

The sensors measurement thread handles the acquisition of data from a heterogeneous set of devices, including:

\begin{itemize}
    \item \textbf{\gls{ADC} sensors:} ambient brightness and soil moisture.
    \item \textbf{\gls{I2C} sensors:} accelerometer, temperature/humidity sensor, and \gls{RGB} color sensor.
\end{itemize}

The thread stores all gathered data in the shared \texttt{system\_measurement} structure using atomic operations to guarantee thread-safe data consistency.

The initialization routine creates the sensors thread and assigns its execution parameters. The thread begins running immediately after creation.

\begin{lstlisting}[language=C, caption={Sensors thread initialization}, label={lst:start_sensors}]
void start_sensors_thread(struct system_context *ctx,
                          struct system_measurement *measure) {

    k_thread_create(&sensors_thread_data,
                    sensors_stack,
                    K_THREAD_STACK_SIZEOF(sensors_stack),
                    sensors_thread_fn,
                    ctx, measure, NULL,
                    SENSORS_THREAD_PRIORITY, 0, K_NO_WAIT);

    k_thread_name_set(&sensors_thread_data, "sensors_thread");
}
\end{lstlisting}

The configuration of the sensors thread includes the definition of its stack, priority, and control block, as shown in \autoref{lst:sensors_config}. Zephyr's \texttt{K\_THREAD\_STACK\_DEFINE} macro is used to statically allocate the execution stack.

\begin{lstlisting}[language=C, caption={Sensors thread configuration}, label={lst:sensors_config}]
#define SENSORS_THREAD_STACK_SIZE 1024
#define SENSORS_THREAD_PRIORITY   5

K_THREAD_STACK_DEFINE(sensors_stack, SENSORS_THREAD_STACK_SIZE);
static struct k_thread sensors_thread_data;
\end{lstlisting}

\gls{ADC}-based sensors (brightness and soil moisture) are processed using the utility function \texttt{read\_adc\_percentage()}, shown in Listing~\ref{lst:adc_read}. This function converts the raw ADC voltage into a scaled percentage value, expressed as percentage times ten to preserve one decimal point of precision.

\begin{lstlisting}[language=C, caption={ADC percentage acquisition helper function}, label={lst:adc_read}]
static void read_adc_percentage(const struct adc_config *cfg, atomic_t *target,
                                const char *label, int32_t *mv)
{
    if (adc_read_voltage(cfg, mv) == 0) {
        int32_t percent10 = ((*mv) * 1000) / cfg->vref_mv;
        atomic_set(target, percent10);
    } else {
        printk("[ADC]: %s read error\n", label);
    }
}
\end{lstlisting}

The accelerometer is interfaced over \gls{I2C} and provides raw XYZ readings which are converted to acceleration values in m/s\(^2\) using the device's full-scale range. The processed values are scaled by 100 to preserve two decimal places of resolution. \autoref{lst:accel_read} shows the implementation of the accelerometer handling routine.

\begin{lstlisting}[language=C, caption={Accelerometer data acquisition}, label={lst:accel_read}]
static void read_accelerometer(const struct i2c_dt_spec *dev, uint8_t range,
                               atomic_t *x_ms2, atomic_t *y_ms2, atomic_t *z_ms2) {
    int16_t x_raw, y_raw, z_raw;
    float x_val, y_val, z_val;

    if (accel_read_xyz(dev, &x_raw, &y_raw, &z_raw) == 0) {
        accel_convert_to_ms2(x_raw, range, &x_val);
        accel_convert_to_ms2(y_raw, range, &y_val);
        accel_convert_to_ms2(z_raw, range, &z_val);

        atomic_set(x_ms2, (int32_t)(x_val * 100));
        atomic_set(y_ms2, (int32_t)(y_val * 100));
        atomic_set(z_ms2, (int32_t)(z_val * 100));
    } else {
        printk("[ACCELEROMETER] - Error reading accelerometer\n");
    }
}
\end{lstlisting}

The temperature and humidity sensor also communicates through \gls{I2C}. Humidity measurement implicitly triggers a temperature conversion, after which the associated temperature value can be read. Both humidity and temperature values are scaled by a factor of 100.

\begin{lstlisting}[language=C, caption={Temperature and humidity acquisition}, label={lst:temp_hum}]
static void read_temperature_humidity(const struct i2c_dt_spec *dev,
                                      atomic_t *temp, atomic_t *hum) {

    float humidity;

    if (temp_hum_read_humidity(dev, &humidity) == 0) {
        float temperature;
        uint8_t buf[2];

        int ret = i2c_write_read_dt(dev,
                                    (uint8_t[]){ TH_READ_TEMP_FROM_RH },
                                    1, buf, 2);

        if (ret == 0) {
            uint16_t raw_temp = ((uint16_t)buf[0] << 8) | buf[1];
            temperature = ((175.72f * raw_temp) / 65536.0f) - 46.85f;
        } else {
            printk("[TEMP_HUM SENSOR] - Error reading temperature from RH (%d)\n", ret);
            return;
        }

        atomic_set(hum,  (int32_t)(humidity * 100));
        atomic_set(temp, (int32_t)(temperature * 100));

    } else {
        printk("[TEMP_HUM SENSOR] - Read error (humidity)\n");
    }
}
\end{lstlisting}

The \gls{RGB} color sensor provides raw red, green, blue, and clear-channel information. These values are written directly into the measurement structure without additional scaling, as shown in \autoref{lst:color_read}.

\begin{lstlisting}[language=C, caption={Color sensor acquisition}, label={lst:color_read}]
static void read_color_sensor(const struct i2c_dt_spec *dev,
                              struct system_measurement *measure) {
    ColorSensorData color_data;

    if (color_read_rgb(dev, &color_data) == 0) {
        atomic_set(&measure->red,   color_data.red);
        atomic_set(&measure->green, color_data.green);
        atomic_set(&measure->blue,  color_data.blue);
        atomic_set(&measure->clear, color_data.clear);
    } else {
        printk("[COLOR SENSOR] - Read error\n");
    }
}
\end{lstlisting}

The main execution loop of the sensors thread is shown below. The thread waits for a semaphore signal before performing a complete acquisition cycle across all sensors. Once finished, it releases a semaphore to notify the main thread that new measurements are available.

\begin{lstlisting}[language=C, caption={Sensors thread main loop}, label={lst:sensors_thread_fn}]
static void sensors_thread_fn(void *arg1, void *arg2, void *arg3) {
    struct system_context *ctx = (struct system_context *)arg1;
    struct system_measurement *measure = (struct system_measurement *)arg2;

    int32_t mv = 0;

    while (1) {
        k_sem_take(ctx->sensors_sem, K_FOREVER);

        read_adc_percentage(ctx->phototransistor, &measure->brightness, "Brightness", &mv);
        read_adc_percentage(ctx->soil_moisture, &measure->moisture, "Moisture", &mv);
        read_accelerometer(ctx->accelerometer, ctx->accel_range,
                           &measure->accel_x_g, &measure->accel_y_g, &measure->accel_z_g);
        read_temperature_humidity(ctx->temp_hum, &measure->temp, &measure->hum);
        read_color_sensor(ctx->color, measure);

        k_sem_give(ctx->main_sensors_sem);
    }
}
\end{lstlisting}

The public interface for the sensors thread is shown in \autoref{lst:sensors_header}. It exposes the initialization function and documents the required input structures.

\begin{lstlisting}[language=C, caption={Sensors thread public header}, label={lst:sensors_header}]
#ifndef SENSORS_THREAD_H
#define SENSORS_THREAD_H

#include "main.h"

void start_sensors_thread(struct system_context *ctx,
                          struct system_measurement *measure);

#endif /* SENSORS_THREAD_H */
\end{lstlisting}


\subsubsection{\acrshort{GPS} Thread}

The \gls{GPS} measurement thread is responsible for interfacing with the \gls{GPS} driver, extracting relevant \gls{NMEA} \gls{GGA} information, and updating the global \texttt{system\_measurement} structure. Its main characteristics include:

\begin{itemize}
    \item Periodic \gls{GPS} polling synchronized with the system's operational mode.
    \item Thread-safe shared-memory updates using atomic setters.
    \item Use of semaphores, thread stacks, and thread control blocks.
    \item Scaled integer representation of latitude, longitude, altitude, and \gls{UTC} time.
\end{itemize}

The \gls{GPS} measurement thread is created and launched through the function \texttt{start\_gps\_thread()}, shown in \autoref{lst:start_thread}. This routine initializes the thread with its designated stack, priority, entry function, and arguments.

\begin{lstlisting}[language=C, caption={Initialization of the \acrshort{GPS} measurement thread}, label={lst:start_thread}]
void start_gps_thread(struct system_context *ctx,
                      struct system_measurement *measure) {

    k_thread_create(&gps_thread_data,
                    gps_stack,
                    K_THREAD_STACK_SIZEOF(gps_stack),
                    gps_thread_fn,
                    ctx, measure, NULL,
                    GPS_THREAD_PRIORITY, 0, K_NO_WAIT);

    k_thread_name_set(&gps_thread_data, "gps_thread");
}
\end{lstlisting}

Memory allocation and priority assignment for the \gls{GPS} thread are specified as shown in \autoref{lst:gps_config}. A dedicated stack is defined using Zephyr's \texttt{K\_THREAD\_STACK\_DEFINE} macro, and a thread control block is declared to manage its execution context.

\begin{lstlisting}[language=C, caption={GPS thread configuration in Zephyr}, label={lst:gps_config}]
#define GPS_THREAD_STACK_SIZE 1024
#define GPS_THREAD_PRIORITY   5

K_THREAD_STACK_DEFINE(gps_stack, GPS_THREAD_STACK_SIZE);
static struct k_thread gps_thread_data;
\end{lstlisting}

The core of the \gls{GPS} data-handling logic is encapsulated in the helper function \texttt{read\_gps\_data()}, shown in \autoref{lst:read_gps}. This function waits for a valid \gls{NMEA} \gls{GGA} frame, extracts geographic coordinates, altitude, satellite count, and \gls{UTC} time, and stores them as scaled integers in the shared measurement structure.

Latitude and longitude are scaled by \(10^6\) to preserve decimal precision, while altitude is scaled by a factor of 100. \gls{UTC} time is encoded in the \texttt{HHMMSS} format as a six-digit integer. Adding 1 to the hour component accounts for timezone adjustment (Spain \gls{UTC}+1).

\begin{lstlisting}[language=C, caption={\acrshort{GPS} data acquisition helper function}, label={lst:read_gps}]
static void read_gps_data(gps_data_t *data,
                          struct system_measurement *measure,
                          struct system_context *ctx) {

    if (gps_wait_for_gga(data, K_MSEC(1000)) == 0) {
        atomic_set(&measure->gps_lat,  (int32_t)(data->lat * 1e6f));
        atomic_set(&measure->gps_lon,  (int32_t)(data->lon * 1e6f));
        atomic_set(&measure->gps_alt,  (int32_t)(data->alt * 100.0f));
        atomic_set(&measure->gps_sats, (int32_t)data->sats);

        if (strlen(data->utc_time) >= 6) {
            int hh = (data->utc_time[0] - '0') * 10 + (data->utc_time[1] - '0') + 1;
            int mm = (data->utc_time[2] - '0') * 10 + (data->utc_time[3] - '0');
            int ss = (data->utc_time[4] - '0') * 10 + (data->utc_time[5] - '0');

            int time_int = hh * 10000 + mm * 100 + ss;
            atomic_set(&measure->gps_time, time_int);
        } else {
            atomic_set(&measure->gps_time, -1);
        }

    } else {
        printk("[GPS] - Timeout or invalid data\n");
    }
}
\end{lstlisting}

The main execution loop of the \gls{GPS} thread is shown in \autoref{lst:gps_thread_fn}. The thread waits for a semaphore signal indicating that a \gls{GPS} reading should be performed. Once awakened, it acquires a new GPS sample and signals the main thread upon completion. This mechanism provides deterministic synchronization between system components.

\begin{lstlisting}[language=C, caption={\acrshort{GPS} thread entry routine}, label={lst:gps_thread_fn}]
static void gps_thread_fn(void *arg1, void *arg2, void *arg3) {
    struct system_context *ctx = (struct system_context *)arg1;
    struct system_measurement *measure = (struct system_measurement *)arg2;

    gps_data_t gps_data = {0};

    while (1) {
        k_sem_take(ctx->gps_sem, K_FOREVER);
        read_gps_data(&gps_data, measure, ctx);
        k_sem_give(ctx->main_gps_sem);
    }
}
\end{lstlisting}

The corresponding public interface is declared in the header file \texttt{gps\_thread.h}. As shown below, it specifies the initialization function and documents the dependency on \texttt{system\_context} and \texttt{system\_measurement} structures.

\begin{lstlisting}[language=C, caption={GPS thread public interface}, label={lst:gps_header}]
#ifndef GPS_THREAD_H
#define GPS_THREAD_H

#include "main.h"

void start_gps_thread(struct system_context *ctx,
                      struct system_measurement *measure);

#endif /* GPS_THREAD_H */
\end{lstlisting}

\clearpage
\subsection{Main}

This section describes the main application logic of the system. It includes global configuration parameters, peripheral initialization, shared data structures, and the control flow that governs the system operating modes. The \textit{Main} module acts as the central coordinator, managing timing, user interaction, and synchronization between threads.

\subsubsection{Macro definitions}

This subsection defines the main compile-time configuration parameters used throughout the application. These macros control system behavior such as operating modes, measurement periods, PWM timing, sensor configuration, and valid measurement ranges. Centralizing these definitions improves maintainability and allows system behavior to be adjusted without modifying the application logic.

\textbf{Main Configuration}

The following macros specify the initial operating mode and the timing parameters associated with each system mode. They also define timer periods for \gls{RGB} \gls{LED} updates and statistical reporting, as well as parameters required for software-based \gls{PWM} generation.

\begin{lstlisting}[language=C, caption={Main configuration macros}]
    #define INITIAL_MODE TEST_MODE  /**< Initial operating mode at startup. */

    #define TEST_PERIOD 2000      /**< Test mode measurement period in milliseconds. */
    #define NORMAL_PERIOD 10000    /**< Normal mode measurement period in milliseconds. */

    #define RGB_TIMER_PERIOD 500 /**< RGB LED timer period in milliseconds. */
    #define STATS_TIMER_PERIOD 60000 /**< Statistics reporting period (ms). */

    #define PWM_STEP    1              /**< PWM step in milliseconds. */
    #define PWM_PERIOD  15             /**< PWM period in milliseconds. */
    #define PWM_STEPS   (PWM_PERIOD / PWM_STEP) /**< Number of PWM steps per period. */

\end{lstlisting}

\textbf{Sensors Configuration}

These macros configure the operating parameters of the connected sensors, including measurement ranges, gain settings, integration times, and resolution. The selected values represent a trade-off between accuracy, response time, and power consumption.

\begin{lstlisting}[language=C, caption={Sensors configuration macros}]
    #define ACCEL_RANGE ACCEL_2G    /**< Accelerometer full-scale range setting. */

    #define COLOR_GAIN GAIN_4X      /**< Color sensor gain setting. */
    #define COLOR_INTEGRATION_TIME INTEGRATION_154MS /**< Color sensor integration time in milliseconds. */ 

    #define TEMP_HUM_RESOLUTION TH_RES_RH12_TEMP14  /**< Temperature and humidity sensor resolution setting. */
\end{lstlisting}

\textbf{Measurement Limits}

The measurement limit macros define the acceptable operating ranges for each sensor. These thresholds are used to detect abnormal conditions and to trigger visual or logical alarms when sensor readings fall outside predefined bounds.

\begin{lstlisting}[language=C, caption={Measurement limits macros}]
    #define TEMP_MIN   -10   /**< Minimum temperature in C. */
    #define TEMP_MAX   50    /**< Maximum temperature in C. */

    #define HUM_MIN    25    /**< Minimum humidity percentage. */
    #define HUM_MAX    75    /**< Maximum humidity percentage. */

    #define LIGHT_MIN  0     /**< Minimum brightness percentage. */
    #define LIGHT_MAX  100   /**< Maximum brightness percentage. */

    #define MOISTURE_MIN   0    /**< Minimum soil moisture percentage. */
    #define MOISTURE_MAX   100  /**< Maximum soil moisture percentage. */

    #define COLOR_MIN     0    /**< Minimum raw color channel value. */
    #define COLOR_MAX     65535 /**< Maximum raw color channel value. */

    #define ACCEL_MIN  -2    /**< Minimum acceleration in g. */
    #define ACCEL_MAX   2    /**< Maximum acceleration in g. */
\end{lstlisting}

\textbf{Flags for limits alarms}

Each flag represents a specific sensor exceeding its allowed range. These bitwise flags allow multiple alarm conditions to be tracked simultaneously using a single atomic variable.

\begin{lstlisting}[language=C, caption={Measurement limits flags macros}]
    #define FLAG_TEMP     (1U << 0)
    #define FLAG_HUM      (1U << 1)
    #define FLAG_LIGHT    (1U << 2)
    #define FLAG_MOISTURE (1U << 3)
    #define FLAG_COLOR    (1U << 4)
    #define FLAG_ACCEL    (1U << 5)
\end{lstlisting}

\subsubsection{Peripheral configuration}

This subsection defines the static configuration structures for all hardware peripherals used by the system. These configurations abstract the hardware details and provide a consistent interface for sensor drivers, communication buses, and user interface components.

Analog peripherals such as the phototransistor and soil moisture sensor are configured through \gls{ADC} channels, while digital sensors communicate via the \gls{I2C} bus. The \gls{GPS} module uses a \gls{UART} interface, and \glspl{LED} and the user button are configured through \gls{GPIO} abstractions.

\begin{lstlisting}[language=C, caption={Peripheral configuration macros}]
    /**
     * @brief Phototransistor ADC configuration.
     */
    static struct adc_config pt = {
        .dev = DEVICE_DT_GET(DT_NODELABEL(adc1)),
        .channel_id = 5,
        .resolution = 12,
        .gain = ADC_GAIN_1,
        .ref = ADC_REF_INTERNAL,
        .acquisition_time = ADC_ACQ_TIME_DEFAULT,
        .vref_mv = 3300,
    };

    /**
     * @brief Soil moisture ADC configuration.
     */
    static struct adc_config sm = {
        .dev = DEVICE_DT_GET(DT_NODELABEL(adc1)),
        .channel_id = 0,
        .resolution = 12,
        .gain = ADC_GAIN_1,
        .ref = ADC_REF_INTERNAL,
        .acquisition_time = ADC_ACQ_TIME_DEFAULT,
        .vref_mv = 3300,
    };

    /**
     * @brief Accelerometer I2C configuration.
     */
    static struct i2c_dt_spec accel = {
        .bus = DEVICE_DT_GET(DT_NODELABEL(i2c2)),
        .addr = ACCEL_I2C_ADDR,
    };

    /**
     * @brief Temperature and humidity sensor I2C configuration.
     */
    static struct i2c_dt_spec th = {
        .bus = DEVICE_DT_GET(DT_NODELABEL(i2c2)),
        .addr = TH_I2C_ADDR,
    };

    /**
     * @brief Color sensor I2C configuration.
     */
    static struct i2c_dt_spec color = {
        .bus = DEVICE_DT_GET(DT_NODELABEL(i2c2)),
        .addr = COLOR_I2C_ADDR,
    };

    /**
     * @brief GPS UART configuration.
     */
    static struct gps_config gps = {
        .dev = DEVICE_DT_GET(DT_NODELABEL(usart1)),
    };

    /**
     * @brief RGB LED bus configuration.
     */
    static struct bus_rgb_led rgb_leds = {
        .pins = {
            GPIO_DT_SPEC_GET(DT_ALIAS(red), gpios),
            GPIO_DT_SPEC_GET(DT_ALIAS(green), gpios),
            GPIO_DT_SPEC_GET(DT_ALIAS(blue), gpios)
        },
        .pin_count = BUS_SIZE,
    };

    /**
     * @brief Indicator LED bus configuration.
     */
    static struct bus_led leds = {
        .pins = {
            GPIO_DT_SPEC_GET(DT_ALIAS(led2), gpios),
            GPIO_DT_SPEC_GET(DT_ALIAS(led1), gpios),
            GPIO_DT_SPEC_GET(DT_ALIAS(led0), gpios)
        },
        .pin_count = BUS_SIZE,
    };

    /**
     * @brief User button configuration.
     */
    static struct user_button button = {
        .spec = GPIO_DT_SPEC_GET(DT_ALIAS(sw0), gpios),
    };
\end{lstlisting}

\subsubsection{Data}

This subsection presents the data structures and enumerations used for system state management and inter-thread communication.

The \texttt{system\_mode\_t} enumeration defines the three operating modes of the system, each corresponding to a different behavior and level of functionality. The dominant color enumeration is used to classify color sensor readings and to drive \gls{RGB} \gls{LED} feedback.

Shared structures such as \texttt{system\_context} and \texttt{system\_measurement} enable safe data exchange between threads. Sensor measurements are stored using atomic variables to guarantee consistency in a concurrent execution environment.

The \texttt{main\_measurement} structure aggregates all processed data required by the main thread, while the statistics structure stores accumulated values used for computing mean, minimum, and maximum measurements over time.

\begin{lstlisting}[language=C, caption={Data structures and enumerations}]
    /**
     * @enum system_mode_t
     * @brief System operating modes.
     *
     * These modes define how the system behaves:
     * - **TEST_MODE:** Shows the dominant detected color through the RGB LED.
     * - **NORMAL_MODE:** Periodically measures sensors and alerts if any reading
     *   is out of range.
     * - **ADVANCED_MODE:** Emulates PWM for RGB LED control.
     */
    typedef enum {
        TEST_MODE = 0,    /**< Test mode - displays the dominant color. */
        NORMAL_MODE,      /**< Normal mode - periodic measurement and alerts. */
        ADVANCED_MODE     /**< Advanced mode - minimal visual feedback. */
    } system_mode_t;

    /**
     * @enum dom_color_t
     * @brief Dominant color types. 
     */
    typedef enum {
        DOM_RED,
        DOM_GREEN,
        DOM_BLUE
    } dom_color_t;

    const char* dom_color_names[] = { "RED", "GREEN", "BLUE" };

    /**
     * @brief Shared system context.
     *
     * The @ref system_context structure holds references to the peripheral
     * configurations. The main, sensor, and GPS threads use this structure
     * to coordinate configuration and mode updates.
     */
    static struct system_context ctx = {
        .phototransistor = &pt,
        .soil_moisture = &sm,
        .accelerometer = &accel,
        .accel_range = ACCEL_RANGE,
        .temp_hum = &th,
        .color = &color,
        .gps = &gps,
        .main_sensors_sem = &main_sensors_sem,
        .main_gps_sem = &main_gps_sem,
        .sensors_sem = &sensors_sem,
        .gps_sem = &gps_sem,
    };

    /**
     * @brief Shared measurements between threads.
     *
     * The @ref system_measurement structure holds the current sensor readings
     * accessible to all threads.
     */
    static struct system_measurement measure = {
        .brightness = ATOMIC_INIT(0),
        .moisture = ATOMIC_INIT(0),
        .accel_x_g = ATOMIC_INIT(0),
        .accel_y_g = ATOMIC_INIT(0),
        .accel_z_g = ATOMIC_INIT(0),
        .temp = ATOMIC_INIT(0),
        .hum = ATOMIC_INIT(0),
        .red = ATOMIC_INIT(0),
        .green = ATOMIC_INIT(0),
        .blue = ATOMIC_INIT(0),
        .clear = ATOMIC_INIT(0),
        .gps_lat = ATOMIC_INIT(0),
        .gps_lon = ATOMIC_INIT(0),
        .gps_alt = ATOMIC_INIT(0),
        .gps_sats = ATOMIC_INIT(0),
        .gps_time = ATOMIC_INIT(0),
    };

    /**
     * @brief Main data measurement structure.
     */
    struct main_measurement {
        system_mode_t mode;
        float light;
        float moisture;
        float lat;
        float lon;
        float alt;
        float x_axis;
        float y_axis;
        float z_axis;
        float hum;
        float temp;
        int sats;
        int time_int;
        int hh, mm, ss;
        float c, r, g, b;
        char ns;
        char ew;
        dom_color_t dom_color;
        atomic_t rgb_flags;
    };

    /**
     * @brief Main measurement data initialization.
     */
    static struct main_measurement main_data = {
        .mode = INITIAL_MODE,
        .light = 0.0f,
        .moisture = 0.0f,
        .lat = 0.0f,
        .lon = 0.0f,
        .alt = 0.0f,
        .x_axis = 0.0f,
        .y_axis = 0.0f,
        .z_axis = 0.0f,
        .hum = 0.0f,
        .temp = 0.0f,
        .sats = 0,
        .time_int = 0,
        .hh = 0,
        .mm = 0,
        .ss = 0,
        .c = 0,
        .r = 0,
        .b = 0,
        .g = 0,
        .ns = '\0',
        .ew = '\0',
        .dom_color = DOM_RED,
        .rgb_flags = ATOMIC_INIT(0),
    };

    /**
     * @brief Statistics data structure for mean, max, and min calculations.
     */
    struct stats_measurements {
        float temp_mean, temp_max, temp_min;
        float hum_mean, hum_max, hum_min;
        float light_mean, light_max, light_min;
        float moisture_mean, moisture_max, moisture_min;
        float x_axis_max, x_axis_min;
        float y_axis_max, y_axis_min;
        float z_axis_max, z_axis_min;
        int red_count, green_count, blue_count;
        int count;
    };

    /**
     * @brief Statistics data initialization.
     */
    struct stats_measurements stats_data = {0};
\end{lstlisting}

\subsubsection{Initialization}

During system startup, all peripherals are initialized sequentially. Each initialization step is validated, and the program execution is halted if any critical component fails to initialize correctly. This approach ensures that the system does not operate in an undefined or partially configured state.

Timers, worker threads, and synchronization primitives are then initialized. Measurement threads for sensors and \gls{GPS} data acquisition are started, and the system enters its initial operating mode with a visual indication provided by the \glspl{LED}.

\begin{lstlisting}[language=C, caption={Initialization function}]
    printk("==== Plant Monitoring System ====\n");
    printk("System ON (TEST MODE)\n\n");

    uint32_t flags = 0;
    system_mode_t previous_mode = INITIAL_MODE;
    float r_norm = 0.0f, g_norm = 0.0f, b_norm = 0.0f;
    int r_duty = 0, g_duty = 0, b_duty = 0, r_value = 0, g_value = 0, b_value = 0;
    bool keep_running = true;

    /* Initialize peripherals */
    if (gps_init(&gps)) {
        printk("GPS initialization failed - Program stopped\n");
        return -1;
    }
    if (adc_init(&pt)) {
        printk("Phototransistor initialization failed - Program stopped\n");
        return -1;
    }
    if (adc_init(&sm)) {
        printk("Soil moisture sensor initialization failed - Program stopped\n");
        return -1;
    }
    if (accel_init(&accel, ACCEL_RANGE)) {
        printk("Accelerometer initialization failed - Program stopped\n");
        return -1;
    }
    if (temp_hum_init(&th, TEMP_HUM_RESOLUTION)) {
        printk("Temperature/Humidity sensor initialization failed - Program stopped\n");
        return -1;
    }
    if (color_init(&color, COLOR_GAIN, COLOR_INTEGRATION_TIME)) {
        printk("Color sensor initialization failed - Program stopped\n");
        return -1;
    }
    if (led_init(&leds) || led_off(&leds)) {
        printk("LED initialization failed - Program stopped\n");
        return -1;
    }
    if (rgb_led_init(&rgb_leds) || rgb_led_off(&rgb_leds)) {
        printk("RGB LED initialization failed - Program stopped\n");
        return -1;
    }
    if (button_init(&button))  {
        printk("Button initialization failed - Program stopped\n");
        return -1;
    }
    if (button_set_callback(&button, button_isr)) {
        printk("Button callback setup failed - Program stopped\n");
        return -1;
    }

    /* Initialize timers */
    k_timer_init(&main_timer, main_timer_handler, NULL);
    k_timer_init(&rgb_timer, rgb_timer_handler, NULL);
    k_timer_init(&stats_timer, stats_timer_handler, NULL);

    k_timer_start(&main_timer, K_MSEC(TEST_PERIOD), K_MSEC(TEST_PERIOD));
    k_timer_start(&stats_timer, K_MSEC(STATS_TIMER_PERIOD), K_MSEC(STATS_TIMER_PERIOD));

    /* Button handling */
    k_work_init(&button_work, button_work_handler);

    /* Start measurement threads */
    start_sensors_thread(&ctx, &measure);
    start_gps_thread(&ctx, &measure);

    blue(&leds);
\end{lstlisting}

\subsubsection{Modes}

The system operates as a finite-state machine with three distinct modes: \textit{Test}, \textit{Normal}, and \textit{Advanced}. Mode transitions are triggered by the user button, while periodic execution within each mode is controlled by a timer.

A semaphore is used to synchronize the main loop with both the main timer and the button interrupt. This mechanism ensures deterministic behavior and prevents race conditions between asynchronous events.

Each mode provides different functionality and board \gls{LED} feedback, allowing the user to easily identify the current operating state of the system.

\begin{lstlisting}
    while (1) {
        switch (main_data.mode) {

            case TEST_MODE:
                blue(&leds);

                ...

                k_sem_take(&main_sem, K_FOREVER);

                break;

            case NORMAL_MODE:
                green(&leds);
                
                ...

                k_sem_take(&main_sem, K_FOREVER);
                
                break;

            case ADVANCED_MODE:
                red(&leds);

                ...
                        if (k_sem_take(&main_sem, K_NO_WAIT) == 0) {
                            keep_running = false;
                            break;
                        }
                ... 

                break;
        }
    }
\end{lstlisting}


\subsubsection{Main timer}

The main timer defines the duration of each operating cycle within a given mode. When the timer expires, its handler releases a semaphore that unblocks the main loop, allowing the next iteration of the mode logic to execute.

\begin{lstlisting}[language=C, caption={Main timer handler function}]
    static struct k_timer main_timer;

    static void main_timer_handler(struct k_timer *timer)
    {
        k_sem_give(&main_sem);
    }
\end{lstlisting}


\subsubsection{Button}

The user button provides manual control over system mode transitions. Button presses are handled through an \gls{ISR}, which delegates processing to a work queue to avoid executing complex logic in interrupt context.

Each button press advances the system to the next operating mode in a cyclic manner. After updating the mode, the semaphore is released to immediately apply the change in the main execution loop.

\begin{lstlisting}[language=C, caption={Button work handler and ISR functions}]
    static void button_work_handler(struct k_work *work)
    {
        switch (main_data.mode) {
            case TEST_MODE:
                main_data.mode = NORMAL_MODE;
                printk("\nNORMAL MODE\n");
                break;
            case NORMAL_MODE:
                main_data.mode = ADVANCED_MODE;
                printk("\nADVANCED MODE\n");
                break;
            case ADVANCED_MODE:
                main_data.mode = TEST_MODE;
                printk("\nTEST MODE\n");
                break;
        }

        k_sem_give(&main_sem);
    }

    static void button_isr(const struct device *dev, struct gpio_callback *cb, uint32_t pins)
    {
        if (!gpio_pin_get_dt(&button.spec)) {
            k_work_submit(&button_work);
        }
    }
\end{lstlisting}

\subsubsection{TEST MODE}

The \textit{Test Mode} is intended for system verification and debugging purposes. In this mode, all sensors are periodically sampled and their raw and processed values are displayed through the serial console. Additionally, the \gls{RGB} \gls{LED} provides immediate visual feedback by indicating the dominant color detected by the color sensor.

When entering this mode, the system reconfigures the main timer to operate with the test period and ensures that any previously active \gls{RGB} timer is stopped. Sensor and \gls{GPS} acquisition threads are explicitly triggered using semaphores, and the main thread waits until all measurements are available before processing them.

The dominant color is determined by comparing the red, green, and blue channel values, and the corresponding \gls{RGB} \gls{LED} color is activated. Execution then blocks on a semaphore, allowing either the main timer or a button event to resume operation.

\begin{lstlisting}[language=C, caption={TEST MODE}]
    case TEST_MODE:
        blue(&leds);

        if(previous_mode != TEST_MODE) {
            k_timer_stop(&rgb_timer);
            rgb_led_off(&rgb_leds);
            k_timer_stop(&main_timer);
            k_timer_start(&main_timer, K_MSEC(TEST_PERIOD), K_MSEC(TEST_PERIOD));
            previous_mode = TEST_MODE;
        }

        k_sem_give(ctx.sensors_sem);
        k_sem_give(ctx.gps_sem);

        k_sem_take(ctx.main_sensors_sem, K_FOREVER);
        k_sem_take(ctx.main_gps_sem, K_FOREVER);

        get_measurements();

        if (main_data.r > main_data.g && main_data.r > main_data.b) {
            rgb_red(&rgb_leds);
            main_data.dom_color = DOM_RED;
        } else if (main_data.g > main_data.r && main_data.g > main_data.b) {
            rgb_green(&rgb_leds);
            main_data.dom_color = DOM_GREEN;
        } else {
            rgb_blue(&rgb_leds);
            main_data.dom_color = DOM_BLUE;
        }

        display_measurements();

        k_sem_take(&main_sem, K_FOREVER);

        break;
\end{lstlisting}

\subsubsection{Get Measurements}

This function retrieves the most recent sensor readings from shared atomic variables and converts them into physical units suitable for further processing and display. Scaling factors are applied to raw values in order to obtain meaningful measurements such as percentages, degrees Celsius, meters per second squared, and geographic coordinates.

\gls{GPS} data is processed to extract latitude, longitude, altitude, satellite count, and time information. Direction indicators (\gls{N}/\gls{S} and \gls{E}/\gls{W}) are derived from the sign of the coordinates, and the time value is decomposed into hours, minutes, and seconds.

All processed values are stored in the main measurement structure, which serves as the central data container for the current system cycle.

\begin{lstlisting}[language=C, caption={Get measurements function}]
    static void get_measurements()
    {
        main_data.moisture = atomic_get(&measure.moisture) / 10.0f;

        main_data.light = atomic_get(&measure.brightness) / 10.0f;

        main_data.lat  = atomic_get(&measure.gps_lat) / 1e6f;
        main_data.lon  = atomic_get(&measure.gps_lon) / 1e6f;
        main_data.alt  = atomic_get(&measure.gps_alt) / 100.0f;
        main_data.sats   = atomic_get(&measure.gps_sats);
        main_data.time_int = atomic_get(&measure.gps_time);

        main_data.ns = (main_data.lat >= 0) ? 'N' : 'S';
        main_data.ew = (main_data.lon >= 0) ? 'E' : 'W';

        main_data.lat = fabsf(main_data.lat);
        main_data.lon = fabsf(main_data.lon);

        if (main_data.time_int >= 0) {
            main_data.hh = main_data.time_int / 10000;
            main_data.mm = (main_data.time_int / 100) % 100;
            main_data.ss = main_data.time_int % 100;
        } else {
            printk("GPS time: --:--:--\n");
        }

        main_data.r = atomic_get(&measure.red);
        main_data.g = atomic_get(&measure.green);
        main_data.b = atomic_get(&measure.blue);
        main_data.c = atomic_get(&measure.clear);

        main_data.x_axis = atomic_get(&measure.accel_x_g) / 100.0f;
        main_data.y_axis = atomic_get(&measure.accel_y_g) / 100.0f;
        main_data.z_axis = atomic_get(&measure.accel_z_g) / 100.0f;

        main_data.temp = atomic_get(&measure.temp) / 100.0f;
        main_data.hum = atomic_get(&measure.hum) / 100.0f;
    }
\end{lstlisting}

\subsubsection{Display Measurements}

This function is responsible for presenting the current system measurements through the serial console. It provides a structured and human-readable output that includes soil moisture, ambient light level, \gls{GPS} information, color sensor readings, accelerometer values, and temperature and humidity data.

\begin{lstlisting}[language=C, caption={Display measurements function}]
    static void display_measurements()
    {
        printk("SOIL MOISTURE: %.1f%%\n", (double)main_data.moisture);

        printk("LIGHT: %.1f%%\n", (double)main_data.light);

        printk("GPS: #Sats: %d Lat(UTC): %.6f %c Long(UTC): %.6f %c Altitude: %.0f m GPS time: %02d:%02d:%02d\n",
                main_data.sats, (double)main_data.lat, main_data.ns, (double)main_data.lon, 
                main_data.ew, (double)main_data.alt, main_data.hh, main_data.mm, main_data.ss);

        printk("COLOR SENSOR: Clear: %.0f Red: %.0f Green: %.0f Blue: %.0f Dominant color: %s \n",
                (double)main_data.c, (double)main_data.r, (double)main_data.g, (double)main_data.b, dom_color_names[main_data.dom_color]);

        printk("ACCELEROMETER: X_axis: %.2f m/s2, Y_axis: %.2f m/s2, Z_axis: %.2f m/s2 \n",
                (double)main_data.x_axis, (double)main_data.y_axis, (double)main_data.z_axis);

        printk("TEMP/HUM: Temperature: %.1fC, Relative Humidity: %.1f%%\n\n",
                (double)main_data.temp, (double)main_data.hum);
    }
\end{lstlisting}

\subsubsection{NORMAL MODE}

The \textit{Normal Mode} represents the standard operating state of the system. In this mode, measurements are performed at a lower frequency than in Test Mode to reduce power consumption while still providing continuous monitoring.

Upon entering Normal Mode, the main timer is reconfigured with the normal measurement period, and an additional \gls{RGB} timer is enabled to handle alarm visualization. Sensor and \gls{GPS} data are acquired synchronously, and the resulting measurements are processed and validated against predefined limits.

Statistical calculations are updated at each cycle (print each hour), and the current measurements are displayed. The system then waits for the next timer expiration or user interaction.

\begin{lstlisting}[language=C, caption={NORMAL MODE}]
    case NORMAL_MODE:
        green(&leds);
        flags = 0;

        if(previous_mode != NORMAL_MODE) {
            k_timer_stop(&main_timer);
            k_timer_start(&main_timer, K_MSEC(NORMAL_PERIOD), K_MSEC(NORMAL_PERIOD));
            k_timer_start(&rgb_timer, K_MSEC(RGB_TIMER_PERIOD), K_MSEC(RGB_TIMER_PERIOD));
            previous_mode = NORMAL_MODE;
        }

        k_sem_give(ctx.sensors_sem);
        k_sem_give(ctx.gps_sem);

        k_sem_take(ctx.main_sensors_sem, K_FOREVER);
        k_sem_take(ctx.main_gps_sem, K_FOREVER);

        get_measurements();

        check_limits(&flags);

        stats_management();
                
        display_measurements();

        k_sem_take(&main_sem, K_FOREVER);
                
        break;
\end{lstlisting}

\subsubsection{Check Limits}

This subsection describes the mechanism used to validate sensor readings against predefined operational thresholds. Each measurement is compared to its corresponding minimum and maximum values, and any out-of-range condition is clipped to the nearest valid limit.

Whenever a limit violation occurs, a corresponding flag is set using a bitwise representation. These flags are stored atomically and later used to drive the \gls{RGB} \gls{LED} alarm system, allowing multiple simultaneous alarm conditions to be represented efficiently.

\begin{lstlisting}[language=C, caption={Check limits function}]
    static void check_limit(float *val, float min, float max, uint32_t *flags, uint32_t flag_bit)
    {
        if (*val < min) {
            *val = min;
            *flags |= flag_bit;
        } else if (*val > max) {
            *val = max;
            *flags |= flag_bit;
        }
    }

    static void check_limits(uint32_t *flags)
    {
        *flags = 0U;

        check_limit(&main_data.temp, TEMP_MIN, TEMP_MAX, flags, FLAG_TEMP);
        check_limit(&main_data.hum, HUM_MIN, HUM_MAX, flags, FLAG_HUM);
        check_limit(&main_data.light, LIGHT_MIN, LIGHT_MAX, flags, FLAG_LIGHT);
        check_limit(&main_data.moisture, MOISTURE_MIN, MOISTURE_MAX, flags, FLAG_MOISTURE);

        check_limit(&main_data.c, COLOR_MIN, COLOR_MAX, flags, FLAG_COLOR);
        check_limit(&main_data.r, COLOR_MIN, COLOR_MAX, flags, FLAG_COLOR);
        check_limit(&main_data.g, COLOR_MIN, COLOR_MAX, flags, FLAG_COLOR);
        check_limit(&main_data.b, COLOR_MIN, COLOR_MAX, flags, FLAG_COLOR);

        check_limit(&main_data.x_axis, ACCEL_MIN * 9.8f, ACCEL_MAX * 9.8f, flags, FLAG_ACCEL);
        check_limit(&main_data.y_axis, ACCEL_MIN * 9.8f, ACCEL_MAX * 9.8f, flags, FLAG_ACCEL);
        check_limit(&main_data.z_axis, ACCEL_MIN * 9.8f, ACCEL_MAX * 9.8f, flags, FLAG_ACCEL);

        atomic_set(&main_data.rgb_flags, (atomic_val_t)(*flags));
    }
\end{lstlisting}

\subsubsection{\acrshort{RGB} Alarms}

The \gls{RGB} alarm system provides visual feedback when one or more sensor measurements exceed their allowed ranges. A dedicated timer with a period of 0.5 seconds cyclically updates the \gls{RGB} \gls{LED}.

Each active alarm condition is mapped to a specific color. When multiple alarms are present, the \gls{LED} cycles through the corresponding colors with a period of 0.5 seconds, ensuring that all active warnings are communicated to the user. If no alarms are active, the \gls{RGB} \gls{LED} is turned off.

\begin{lstlisting}[language=C, caption={RGB LED alarm timer handler}]
    static struct k_timer rgb_timer;

    static void rgb_timer_handler(struct k_timer *timer)
    {
        static uint8_t color_index = 0;
        uint32_t flags = atomic_get(&main_data.rgb_flags);

        uint8_t colors[6]; 
        uint8_t count = 0;

        if (flags & FLAG_TEMP)      colors[count++] = 0; // RED
        if (flags & FLAG_HUM)       colors[count++] = 1; // BLUE
        if (flags & FLAG_LIGHT)     colors[count++] = 2; // GREEN
        if (flags & FLAG_MOISTURE)  colors[count++] = 3; // CYAN
        if (flags & FLAG_COLOR)     colors[count++] = 4; // WHITE
        if (flags & FLAG_ACCEL)     colors[count++] = 5; // YELLOW

        if (count == 0) {
            rgb_led_off(&rgb_leds);
            color_index = 0;
            return;
        }

        uint8_t color_status = colors[color_index % count];
        color_index++;

        switch (color_status) {
            case 0: rgb_red(&rgb_leds); break;
            case 1: rgb_blue(&rgb_leds); break;
            case 2: rgb_green(&rgb_leds); break;
            case 3: rgb_cyan(&rgb_leds); break;
            case 4: rgb_white(&rgb_leds); break;
            case 5: rgb_yellow(&rgb_leds); break;
            default: rgb_led_off(&rgb_leds); break;
        }
    }
\end{lstlisting}

\subsubsection{Stats Management}

This subsection describes the statistical processing performed on sensor data during Normal Mode operation. The system maintains running statistics, including mean, maximum, and minimum values, for selected measurements.

The mean values are computed incrementally to minimize memory usage and computational overhead. Maximum and minimum values are updated dynamically as new measurements become available. Additionally, the system tracks the frequency of dominant color detections to determine long-term color trends.

\begin{lstlisting}[language=C, caption={Stats Management}]
    static void stats_management()
    {
        stats_data.count++;

        mean_calculation();

        max_min_calculation();

        dominant_color_calculation();
    }

    static void mean_calculation()
    {
        if (stats_data.count == 1) {
            stats_data.temp_mean = main_data.temp;
            stats_data.hum_mean = main_data.hum;
            stats_data.light_mean = main_data.light;
            stats_data.moisture_mean = main_data.moisture;
        } else {
            stats_data.temp_mean = ((stats_data.temp_mean * (stats_data.count - 1)) + main_data.temp) / stats_data.count;
            stats_data.hum_mean = ((stats_data.hum_mean * (stats_data.count - 1)) + main_data.hum) / stats_data.count;
            stats_data.light_mean = ((stats_data.light_mean * (stats_data.count - 1)) + main_data.light) / stats_data.count;
            stats_data.moisture_mean = ((stats_data.moisture_mean * (stats_data.count - 1)) + main_data.moisture) / stats_data.count;
        }
    }

    static void max_min_calculation()
    {
        if (stats_data.count == 1) {
            // Initialize first values
            stats_data.temp_max = stats_data.temp_min = main_data.temp;
            stats_data.hum_max = stats_data.hum_min = main_data.hum;
            stats_data.light_max = stats_data.light_min = main_data.light;
            stats_data.moisture_max = stats_data.moisture_min = main_data.moisture;

            stats_data.x_axis_max = stats_data.x_axis_min = main_data.x_axis;
            stats_data.y_axis_max = stats_data.y_axis_min = main_data.y_axis;
            stats_data.z_axis_max = stats_data.z_axis_min = main_data.z_axis;
        } else {
            // Temperature
            if (main_data.temp > stats_data.temp_max) stats_data.temp_max = main_data.temp;
            if (main_data.temp < stats_data.temp_min) stats_data.temp_min = main_data.temp;

            // Humidity
            if (main_data.hum > stats_data.hum_max) stats_data.hum_max = main_data.hum;
            if (main_data.hum < stats_data.hum_min) stats_data.hum_min = main_data.hum;

            // Light
            if (main_data.light > stats_data.light_max) stats_data.light_max = main_data.light;
            if (main_data.light < stats_data.light_min) stats_data.light_min = main_data.light;

            // Moisture
            if (main_data.moisture > stats_data.moisture_max) stats_data.moisture_max = main_data.moisture;
            if (main_data.moisture < stats_data.moisture_min) stats_data.moisture_min = main_data.moisture;

            // Accelerometer
            if (main_data.x_axis > stats_data.x_axis_max) stats_data.x_axis_max = main_data.x_axis;
            if (main_data.x_axis < stats_data.x_axis_min) stats_data.x_axis_min = main_data.x_axis;

            if (main_data.y_axis > stats_data.y_axis_max) stats_data.y_axis_max = main_data.y_axis;
            if (main_data.y_axis < stats_data.y_axis_min) stats_data.y_axis_min = main_data.y_axis;

            if (main_data.z_axis > stats_data.z_axis_max) stats_data.z_axis_max = main_data.z_axis;
            if (main_data.z_axis < stats_data.z_axis_min) stats_data.z_axis_min = main_data.z_axis;
        }
    }

    static void dominant_color_calculation()
    {
        if (main_data.r > main_data.g && main_data.r > main_data.b) {
            stats_data.red_count++;
        } else if (main_data.g > main_data.r && main_data.g > main_data.b) {
            stats_data.green_count++;
        } else if (main_data.b > main_data.r && main_data.b > main_data.g) {
            stats_data.blue_count++;
        }
    }
\end{lstlisting}

\subsubsection{ADVANCED MODE}

Explained in \autoref{adv_mode_section}.
\clearpage
\subsection{Zephyr \acrshort{RTOS}} 

\subsubsection{prj\_nucleo\_wl55jc.conf}

The following configuration file (\texttt{prj\_nucleo\_wl55jc.conf}) specifies the system modules required for enabling the serial console, \gls{GPIO}, \gls{ADC}, and \gls{I2C} interfaces within the Zephyr \gls{RTOS} environment. It also activates several debugging and runtime analysis features, including thread stack initialization, thread information reporting, and the automatic thread analyzer.

\begin{lstlisting}[caption={prj\_nucleo\_wl55jc.conf}]
    CONFIG_STDOUT_CONSOLE=y
    CONFIG_UART_CONSOLE=y
    CONFIG_CONSOLE=y
    CONFIG_PRINTK=y
    CONFIG_CBPRINTF_FP_SUPPORT=y
    CONFIG_POLL=y

    CONFIG_EVENTS=y
    CONFIG_LOG=y

    CONFIG_GPIO=y # Enable GPIO

    CONFIG_ADC=y  # Enable ADC

    CONFIG_I2C=y  # Enable I2C

    CONFIG_SERIAL=y
    CONFIG_UART_INTERRUPT_DRIVEN=y # Enable UART interrupt-driven API

    CONFIG_INIT_STACKS=y
    CONFIG_THREAD_STACK_INFO=y
    CONFIG_THREAD_ANALYZER=y
    CONFIG_THREAD_ANALYZER_AUTO=y
    CONFIG_THREAD_NAME=y
\end{lstlisting}

\subsubsection{nucleo\_wl55jc.overlay}

Additionally, the DeviceTree overlay file (\texttt{nucleo\_wl55jc.overlay}) extends the hardware description of the Nucleo-WL55JC board by defining a \gls{RGB} \gls{LED} structure implemented through \gls{GPIO}-controlled \gls{LED} nodes. Corresponding aliases are included to simplify application-level access to these components. The overlay also configures the \gls{USART}1 peripheral with its associated pin assignments and baud rate, enabling serial communication capabilities required by the console or other \gls{UART}-based interfaces.

\begin{lstlisting}[caption={nucleo\_wl55jc.overlay}]
    #include <zephyr/dt-bindings/pinctrl/stm32-pinctrl.h>

    / {
        rgb_leds {
            compatible = "gpio-leds";

            rgb_red: rgb_0 {
                gpios = <&gpioa 6 GPIO_ACTIVE_LOW>;
                label = "Red RGB LED";
            };
            rgb_green: rgb_1 {
                gpios = <&gpioa 7 GPIO_ACTIVE_LOW>;
                label = "Green RGB LED";
            };
            rgb_blue: rgb_2 {
                gpios = <&gpioa 9 GPIO_ACTIVE_LOW>;
                label = "Blue RGB LED";
            };
        };

        aliases {
            red = &rgb_red;
            green = &rgb_green;
            blue = &rgb_blue;
            led0 = &blue_led_1; 	// This is LED1 as labeled STM32WL55JC board's 
    		led1 = &green_led_2; 	// This is LED2 as labeled STM32WL55JC board's 
    		led2 = &red_led_3; 	    // This is LED3 as labeled STM32WL55JC board's 
        };


    };

    &usart1 {
        status = "okay";
        current-speed = <9600>;
        pinctrl-0 = <&usart1_tx_pb6 &usart1_rx_pb7>;
        pinctrl-names = "default";
    };

\end{lstlisting}


\subsection{Thread Stack and \acrshort{CPU} Usage Analysis}

Zephyr provides runtime diagnostics that allow monitoring of the stack usage and \gls{CPU} load of each thread in the system. The output shown in the image presents detailed information for all active threads, including their stack consumption, remaining free stack space, and the total number of \gls{CPU} cycles executed since startup.

\begin{figure}[H]
    \centering
    \includegraphics[width=0.8\textwidth]{images/stack.png}
    \caption{Thread stack and \gls{CPU} usage}
    \label{fig:stack}
\end{figure}

It is important to understand the purpose of each thread displayed:

\begin{itemize}
    \item \textbf{gps\_thread:} Thread responsible for configuring, reading and parsing \gls{GPS} data.

    \item \textbf{sensors\_thread:} Thread responsible for sensor readings (accelerometer, colour sensor, etc.).

    \item \textbf{thread\_analyzer:} Internal diagnostic thread used to collect and report thread metrics such as stack usage. It runs periodically and only consumes \gls{CPU} during short analysis windows.

    \item \textbf{sysworkq:} The global Zephyr system workqueue. It is used to run small background tasks that do not need their own dedicated thread. Typical examples include executing callbacks.  

    \item \textbf{logging:} Internal Zephyr thread in charge of processing log messages.

    \item \textbf{idle:} Lowest-priority thread that runs whenever no other thread is ready. It accounts for the majority of \gls{CPU} cycles, which is expected and desirable in a low-power sensor system.

    \item \textbf{main:} The initial thread created at system startup.

    \item \textbf{\gls{ISR}0 (\gls{ISR} stack):} Not a regular thread, but the shared stack region used by all interrupt service routines.
\end{itemize}

For each thread, the following metrics are displayed:

\begin{itemize}
    \item \textbf{STACK:} Reports the unused stack space, the amount of stack used, and the total allocated stack size.  
    For example, for \texttt{gps\_thread}:
    \[
    \text{unused } 768 \text{ B},\quad \text{used } 256 \text{ B},\quad \text{total } 1024 \text{ B}
    \]
    This corresponds to a stack usage of 25\%, indicating that the assigned memory is sufficient and no overflow risk is present. 

    As a general guideline, \textbf{stack usage below 60\% is considered safe} in Zephyr, as it leaves enough headroom for context switching, interrupts, and occasional peak loads.

    \item \textbf{\gls{CPU}:} Shows the percentage of \gls{CPU} time consumed by each thread.  
    Most application threads such as \texttt{gps\_thread} or \texttt{sensors\_thread} show 0\% \gls{CPU} usage because they predominantly sleep while waiting for periodic timers or I/O events.

    \item \textbf{Total \gls{CPU} cycles used:} Indicates the cumulative processor cycles consumed by each thread since boot.  
    
    Threads like \texttt{idle} present extremely large values, which is expected since the idle thread runs whenever no other thread is ready to execute. A high idle count is a positive indicator of energy efficiency.
\end{itemize}

This is obtained thanks to the following configuration options enabled in \texttt{prj\_nucleo\_wl55jc.conf}, which allow Zephyr to track stack usage, assign human-readable thread names, and automatically generate periodic thread analysis reports:

\begin{lstlisting}[caption={Thread stack and CPU usage report - prj\_nucleo\_wl55jc.conf}]
    CONFIG_INIT_STACKS=y
    CONFIG_THREAD_STACK_INFO=y
    CONFIG_THREAD_ANALYZER=y
    CONFIG_THREAD_ANALYZER_AUTO=y
    CONFIG_THREAD_NAME=y    
\end{lstlisting}

Overall, the reported values confirm that:

\begin{itemize}
    \item All thread stacks remain within safe usage ranges, with most below the recommended 60\% threshold.
    \item \gls{CPU} usage distribution behaves as expected for a sensor-driven, event-based embedded application.
    \item The idle thread dominates \gls{CPU} cycles, indicating efficient low-power execution and minimal background processing overhead.
\end{itemize}



\subsection{Compilation and Flashing Output Analysis}

During the compilation process, Zephyr generates a memory usage summary that indicates how much Flash and \gls{RAM} the final application occupies. As shown in \autoref{fig:build}, after linking the executable \texttt{zephyr.elf}, the memory report provides the following information:

\begin{itemize}
    \item \textbf{FLASH:} 59.024B used out of 256KB (approximately 22.5\%).
    \item \textbf{\gls{RAM}:} 13.504B used out of 64KB (approximately 20.6\%).
\end{itemize}

This confirms that the firmware comfortably fits within the memory limits of the STM32WL55 microcontroller, leaving sufficient headroom for future improvements or additional functionality.

\begin{figure}[H]
    \centering
    \includegraphics[width=0.8\textwidth]{images/build.png}
    \caption{Compilation memory usage report}
    \label{fig:build}
\end{figure}

\subsection{Flashing the Firmware onto the STM32WL55}

The \autoref{fig:flash} corresponds to the flashing process performed using \texttt{STM32CubeProgrammer}, which communicates with the NUCLEO-WL55JC board via the onboard ST-LINK debugger. The tool successfully identifies the target device, displaying key details such as:

\begin{itemize}
    \item \textbf{Device:} STM32WLxx.
    \item \textbf{Flash Size:} 256KB.
    \item \textbf{Core:} \gls{ARM} Cortex-M4.
    \item \textbf{Supply Voltage:} 3.28V.
    \item \textbf{Connection Mode:} Under Reset.
\end{itemize}

After loading the generated \texttt{zephyr.hex} file (57.64KB), the programmer performs the following steps:

\begin{enumerate}
    \item Erases the internal Flash sectors (0 to 28).
    \item Programs the firmware at address \texttt{0x08000000}.
    \item Verifies the integrity of the written data.
    \item Starts the application.
\end{enumerate}

The final message, \textit{``Application is running, Please Hold on...''}, indicates that the microcontroller has successfully been programmed and is now executing the uploaded Zephyr firmware.


\begin{figure}[H]
    \centering
    \includegraphics[width=0.8\textwidth]{images/flash.png}
    \caption{Flashing process using STM32CubeProgrammer}
    \label{fig:flash}
\end{figure}

\subsection{Code Documentation}

The project documentation is generated automatically through a continuous integration workflow implemented using a GitHub Action (\autoref{ap:github}). This workflow executes the Doxygen engine, which extracts structured information directly from the annotated comments within the source code. By following Doxygen's documentation conventions, each module, function, and data structure is described where it is implemented, ensuring that the documentation remains consistent with the evolving codebase.

Whenever new commits are pushed to the repository, the GitHub Action is triggered, automatically regenerating the documentation and preventing discrepancies between the implementation and its technical description. As part of the same workflow, the generated documentation is automatically deployed to a GitHub Pages site, making it accessible online without requiring manual intervention.

The documentation can be accessed directly through the following link: \url{https://estelamb.github.io/Embedded_IoT/}.
