\section{Project Specifications}\todo{revisar}

\subsection{System Requirements and Infrastructure}

The objective of the system is to acquire environmental data from a plant and transmit it to a \gls{LoRaWAN} network server using the integrated radio of the STM32WL55 platform. The infrastructure relies on a star-of-stars topology consisting of the following elements:

\begin{itemize}
    \item \textbf{\gls{LoRaWAN} Nodes}: Hardware based on the STM32WL55JC microcontroller equipped with environmental and health sensors.
    \item \textbf{Gateways}: A \gls{LoRaWAN} Gateway located on the roof of Building 8 at the university campus. It operates in \textit{Packet Forwarder} mode, relaying messages to the server address \texttt{eu72udp.resiot.io}.
    \item \textbf{Network and Application Server}: The ResIOT.io platform is used for device management, data decoding via LUA scripts, and dashboard visualization.
\end{itemize}



\subsection{Hardware Specifications}

The system utilizes the STM32WL55JC long-range wireless and ultra-low-power device. \autoref{tab:hardware_summary} summarizes the sensors and interfaces used for monitoring all required parameters related to the plant's health.

\begin{table}[H]
    \centering
    \caption{Summary of Hardware for the \gls{LoRaWAN} \gls{IoT} System}
    \label{tab:hardware_summary}
    \begin{tabular}{p{0.30\textwidth}p{0.30\textwidth}p{0.25\textwidth}}
        \toprule
        \textbf{Parameter} & \textbf{Sensor / Module} & \textbf{Interface}  \\ 
        \midrule
        \gls{MCU} Board & STM32WL55JC & \gls{LoRaWAN} (Internal) \\ 
        Status / Downlink & \gls{RGB} \gls{LED} & \gls{GPIO} \\ 
        Ambient Light & Analog Light Sensor & \gls{ADC} \\ 
        Soil Moisture & Analog Moisture Probe & \gls{ADC} \\ 
        Temp / Humidity & \gls{I2C} Environmental Sensor & \gls{I2C} \\ 
        Leaf Colour & \gls{RGB} Colour Sensor & \gls{I2C} \\ 
        Accelerometer & 3-axis Accelerometer & \gls{I2C} \\ 
        Global Location & External \gls{GPS} Module & \gls{UART} \\ 
        \bottomrule
    \end{tabular}
\end{table}

\subsection{Software and Connectivity Requirements}

The project development is structured into three phases, evolving from basic \gls{LoRaWAN} connectivity to a fully optimized monitoring application.

\subsubsection{\gls{LoRaWAN} Configuration}
The node is configured using \textbf{\gls{OTAA}}. The following identifiers are implemented in the \texttt{main.c} file to establish the connection:
\begin{itemize}
    \item \textbf{Device \gls{EUI}}: Unique identifier for every node.
    \item \textbf{Join \gls{EUI} (App \gls{EUI})}: \texttt{0x70B3D57ED000FC4D}.
    \item \textbf{App Key}: Secret key for secure network join.
\end{itemize}

\subsubsection{Data Transmission and Optimization (Phase 3)}
As per Phase 3 requirements, the system optimizes the data format using binary types to minimize message length.

\begin{table}[H]
    \centering
    \caption{Connectivity and Payload Specifications}
    \label{tab:connectivity_specs}
    \begin{tabular}{p{0.25\textwidth}p{0.65\textwidth}}
        \toprule
        \textbf{Feature} & \textbf{Requirement} \\
        \midrule
        Transmission Interval & 60 seconds (defined by \texttt{DELAY}) \\
        Max Message Length & 30 bytes \\
        Implemented Payload & 30 bytes (\texttt{packed} structure) \\
        Activation Mode & \gls{OTAA} \\
        Downlink Control & Remote \gls{LED} change via "OFF", "Green", or "Red" \\
        \bottomrule
    \end{tabular}
\end{table}

\subsection{Additional Specifications Implemented}

In addition to the core requirements, several robustness mechanisms were implemented to improve system reliability:

\begin{table}[H]
    \centering
    \caption{Additional Specifications Implemented}
    \label{tab:additional_specs_lora}
    \begin{tabular}{p{0.05\textwidth}p{0.20\textwidth}p{0.65\textwidth}}
        \toprule
        \textbf{ID} & \textbf{Requirement} & \textbf{Description} \\ 
        \midrule
        AS1 & \textbf{Join Resiliency} & The system implements up to 30 join retries with a delay between attempts to handle gateway saturation. \\ 
        AS2 & \textbf{Atomic Storage} & Shared measurements use Zephyr atomic variables to prevent race conditions between sensor threads and the main \gls{LoRaWAN} loop. \\ 
        AS3 & \textbf{LUA Decoding} & A custom LUA script on the ResIOT server parses the binary payload into human-readable variables for the dashboard. \\ 
        AS4 & \textbf{\gls{GPS} Fail-safe} & Predefined coordinates are used if the \gls{GPS} signal is not fixed, ensuring the transmission loop continues. \\ 
        \bottomrule
    \end{tabular}
\end{table}