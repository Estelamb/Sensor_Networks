\section{System Implementation}

\subsection{End Node}

The End Node is a core component of the sensor network, responsible for environmental data acquisition and event detection. In this project, the node is optimized for autonomous operation using a battery, which necessitates a strict power management strategy.

\subsubsection{Functional Specifications}
The End Node is equipped with a sensor board and an XBee 802.15.4 communication module. Its primary functions are:
\begin{itemize}
    \item \textbf{Periodic Sensing}: Measures temperature, humidity, pressure, and battery level every 30 seconds.
    \item \textbf{Inertial Monitoring}: Constant monitoring of the 3-axis accelerometer to detect free-fall events.
    \item \textbf{Motion Detection}: Integration with a \gls{PIR} sensor to identify human presence.
    \item \textbf{Power Management}: Utilization of deep sleep modes to minimize current consumption during idle periods.
\end{itemize}

\subsubsection{Logic and Operational Flow}
The node operates in a cyclic manner. Upon startup, it initializes the hardware, stabilizes the \gls{PIR} sensor, and joins the XBee network. It then enters a \texttt{deepSleep} state, from which it can be awakened by three specific triggers:
\begin{enumerate}
    \item \textbf{\gls{RTC} Alarm}: A timer-based wake-up every 30 seconds for periodic reporting.
    \item \textbf{Accelerometer Interrupt}: Triggered immediately if the $G$-force threshold for a free fall is met.
    \item \textbf{Sensor Event Interrupt}: Triggered by the \gls{PIR} sensor when motion is detected.
\end{enumerate}

\subsubsection{Flow Diagram}
The following diagram describes the internal logic of the \texttt{loop()} function:

\begin{figure}[H]
    \centering
    \includegraphics[width=0.5\textwidth]{images/end_node.png}
    \caption{End Node Loop Diagram}
    \label{fig:end_node_loop}
\end{figure}

\subsubsection{Code Implementation}
The implementation relies on the \texttt{WaspFrame} library to encapsulate data and the \texttt{PWR} library for energy savings. A critical part of the code is the interrupt handling, which ensures that alarms are sent before the node returns to sleep.

\begin{lstlisting}[language=C, caption={Power Management and Wake-up Logic}]
void loop() {
  // Check for Alarms first
  if (intFlag & (ACC_INT | SENS_INT)) handleAlarms();
  
  // Check for Periodic measurement
  if (intFlag & RTC_INT) {
    intFlag &= ~(RTC_INT); // Clear flag
    readSensors();
    sendPeriodicData();
  }
  
  // Return to low power mode
  PWR.deepSleep(SAMPLING_PERIOD, RTC_OFFSET, RTC_ALM1_MODE1, ALL_OFF);
}
\end{lstlisting}

\subsubsection{Payload Structure}
To ensure the Gateway can parse the information correctly, a key-value format is used. For example, a periodic packet looks like:
\texttt{node\_01:T=\%s;H=\%s;P=\%s;B=\%d;X=\%s;Y=\%s;Z=\%s;}. This format allows for easy string manipulation at the Gateway and ensures that individual parameters can be mapped to specific ThingSpeak fields.

\subsection{Gateway Node}

The Gateway node acts as the critical interface between the \gls{IEEE} 802.15.4 sensor network and the Internet. Unlike the End Nodes, the Gateway is designed to be grid-connected, allowing it to remain in a constant state of wakefulness to process incoming data and maintain a persistent connection to the cloud.

\subsubsection{System Role and Responsibilities}
The Gateway performs three main concurrent tasks:
\begin{itemize}
    \item \textbf{XBee Reception}: It acts as a receiver for all End Node transmissions (periodic data and alarms) within the \gls{IEEE} 802.15.4 network.
    \item \textbf{Local Monitoring}: Like the End Node, it samples its own environment every 30 seconds to provide a baseline for the gateway location.
    \item \textbf{\acrshort{MQTT} Bridge}: It translates the raw string data received via XBee into formatted \acrshort{MQTT} packets to be published to the ThingSpeak broker.
\end{itemize}

\subsubsection{Network Integration and Connectivity}
The Gateway utilizes two communication modules simultaneously:
\begin{enumerate}
    \item \textbf{XBee 802.15.4}: Configured with a matching \gls{PAN ID} (\texttt{0x1234}) and Channel (\texttt{0x0F}) to communicate with the End Nodes.
    \item \textbf{WiFi PRO}: Connects to a local Access Point to provide IP connectivity. The implementation includes a robust \texttt{wifiCheckConnection()} function to ensure the system automatically recovers from network drops.
\end{enumerate}

\subsubsection{Data Parsing and Logic Flow}
A significant portion of the Gateway's logic is dedicated to parsing the incoming XBee strings. Since the data arrives in a raw format (e.g., \texttt{T=24.5;H=48.2...}), the Gateway uses the \texttt{strtok()} function to tokenize and extract individual sensor values before re-packaging them for \gls{MQTT}.

The following diagram illustrates the flow from a remote sensor trigger to the cloud update:

\begin{figure}[H]
    \centering
    \includegraphics[width=1\textwidth]{images/gateway.png}
    \caption{Gateway Sequence Diagram}
    \label{fig:gateway}
\end{figure}

\subsubsection{Code Implementation: The \acrshort{MQTT} Bridge}
The Gateway implements a specific \texttt{mqttPublish} function that manages \gls{TCP} socket creation, \gls{MQTT} packet serialization (Connect and Publish), and socket closure to prevent resource exhaustion.

\begin{lstlisting}[language=C, caption={Gateway Bridge Logic}]
void loop() {
  wifiCheckConnection(); // Ensure IP is active
  
  receiveXBee(); // Listen for remote nodes (500ms timeout)
  
  if (intFlag & RTC_INT) { // Periodic Local Sensing
    readSensors();
    mqttPublish(MQTT_TOPIC_GATEWAY, gateway_temp, gateway_humd, ...);
    RTC.setAlarm1(SAMPLING_PERIOD, RTC_OFFSET, RTC_ALM1_MODE1);
  }
}
\end{lstlisting}

\subsection{ThingSpeak Cloud Configuration}

The visualization and data management layer is implemented using the ThingSpeak cloud platform. Following the system requirements, the information is organized into three distinct channels to optimize data throughput and ensure clear separation between sensor nodes and critical alerts.

\subsubsection{Channel Architecture}
As seen in the platform configuration, the system utilizes three private channels, each updated via \gls{MQTT}:
\begin{itemize}
    \item \textbf{End Node (\gls{ID}: 3256681)}: Dedicated to the remote sensor data received via XBee.
    \item \textbf{Gateway Node (\gls{ID}: 3256684)}: Used for the local environmental data collected by the gateway itself.
    \item \textbf{Alarms (\gls{ID}: 3256686)}: A specialized channel for high-priority event visualization.
\end{itemize}

\subsubsection{Field Mapping and Visualization}
Each channel is configured with seven active fields to represent the full suite of sensor data. The Node channels settings serve as the template for periodic data:
\begin{table}[H]
  \centering
  \caption{ThingSpeak Field Mapping for Periodic Data}
  \begin{tabular}{p{0.1\textwidth}p{0.3\textwidth}p{0.1\textwidth}p{0.3\textwidth}}
    \toprule
    \textbf{Field} & \textbf{Parameter} & \textbf{Field} & \textbf{Parameter} \\ 
    \midrule
    Field 1 & Temperature ($^{\circ}$C) & Field 5 & X-Axis Acceleration ($m/s^2$) \\ 
    Field 2 & Humidity (\%) & Field 6 & Y-Axis Acceleration ($m/s^2$) \\ 
    Field 3 & Pressure (Pa) & Field 7 & Z-Axis Acceleration ($m/s^2$) \\ 
    Field 4 & Battery Level (\%) & & \\ 
    \bottomrule
  \end{tabular}
\end{table}

Data is visualized through a combination of \texttt{Gauge widgets} for real-time status and \texttt{Historical Line Charts} to track environmental evolution over time. Additionally, a \texttt{Map widget} is used to display the static location of the nodes.


\subsubsection{Alarm Prioritization}
The "Alarms" channel is designed for rapid identification of critical events. It uses binary status indicators:
\begin{itemize}
    \item \textbf{Visual Indicators}: Large red status lamps (widgets) turn bright when a Free Fall or Motion event is active.
    \item \textbf{Historical Logging}: Line charts track the frequency and timing of alarms for both the End Node and Gateway.
\end{itemize}

\subsubsection{\acrshort{MQTT} Device Management}
To facilitate secure \gls{M2M} communication, a "Gateway" device has been authorized within the ThingSpeak \gls{MQTT} settings. This device is granted both \texttt{publish} and \texttt{subscribe} permissions across all three channels, utilizing a unique Client \gls{ID} and a Password to authenticate every transmission. Also, a "Phone" device is created with \texttt{subscribe} permissions to allow for mobile application integration, enabling real-time monitoring.

\subsection{Mobile Application Integration}

To provide a portable, real-time monitoring solution, the system includes a mobile application interface developed using the \texttt{\acrshort{IoT} \gls{MQTT} Panel} application\cite{IoTMQTTPanel}. This integration allows for the visualization of sensor data and immediate notification of alarms directly on a smartphone, facilitating remote supervision without the need for a desktop browser.

\subsubsection{Mobile Broker and Client Configuration}
The application acts as a \gls{MQTT} client. For this project, a specific \gls{MQTT} device named \textbf{"Phone"} was authorized within the ThingSpeak cloud settings. The connection is established using the following parameters and credentials:
\begin{itemize}
    \item \textbf{Broker Host}: \texttt{mqtt3.thingspeak.com}.
    \item \textbf{Port}: \texttt{1883}.
    \item \textbf{\gls{MQTT} Client \gls{ID}}: \texttt{IzMtDSU5ExotMwscGD0mOj0}.
    \item \textbf{Authorized Access}: The device is granted both \texttt{publish} and \texttt{subscribe} permissions for the End Node, Gateway, and Alarms channels to ensure full bidirectional data flow capability.
\end{itemize}

\subsubsection{Subscription Topics and Data Harvesting}
The mobile application is configured to bypass the default dashboard prefix to use ThingSpeak's field-specific subscription architecture. The data is retrieved using the following topic format:
\begin{center}
    \texttt{channels/<ChannelID>/subscribe/fields/field<FieldNumber>}
\end{center}
For example, as shown in the application configuration for the Gateway node's temperature, the client subscribes to: \texttt{channels/3256684/subscribe/fields/field1}. This method ensures that the mobile interface updates only when specific data points change, reducing unnecessary battery and data consumption on the mobile device.

\subsubsection{Dashboard Interface Design}
The interface is structured into three primary tabs for organized data management: \texttt{Principal}, \texttt{End Node}, and \texttt{Gateway}.

\textbf{Sensor Monitoring (End Node and Gateway Tabs)}
Each node has a dedicated view featuring specialized widgets for different data types:
\begin{itemize}
    \item \textbf{Analog Gauges}: Used for Temperature (e.g., $23.25^{\circ}$C) and Humidity (e.g., $54.14\%$) to provide immediate visual context.
    \item \textbf{Linear Progress Bars}: Employed to track battery levels, showing the current charge status (e.g., 83\% for the End Node and 100\% for the Gateway).
    \item \textbf{Digital Displays}: Show high-precision atmospheric pressure values in Pascals (e.g., $93861$ Pa).
    \item \textbf{Time-Series Charts}: A multi-variable accelerometer graph plots the X, Y, and Z axes (e.g., Z constant at $\sim 9.8$ $m/s^2$ while static) to monitor inertial stability.
\end{itemize}

\textbf{Alarm Management}
The \textbf{Principal} tab serves as a high-priority notification center. It utilizes large "Lamp" indicator widgets that represent the status of critical events:
\begin{itemize}
    \item \textbf{Dynamic Feedback}: Indicators remain gray (inactive) during normal operation.
    \item \textbf{Alarm Trigger}: Upon detection of a "Free Fall" or "Motion" event, the corresponding lamp turns bright red (e.g., \texttt{Free Fall - End Node} active).
    \item \textbf{Latency}: The mobile app reflects these states within seconds of the physical trigger on the Waspmote hardware.
\end{itemize}